\documentclass{book}
\usepackage{commeunjeustyle}

\begin{document}
\chapter*{Vecteurs}%label=exercices

\section{Translations - Vecteurs associés}
\begin{Exercice}[Translation d'un triangle]%link=Vecteurs#Vecteurs,type=application,seance=Vecteur associé à une translation
 Reproduire la figure puis construire le translaté du triangle $ABC$ dans la translation de vecteur $\Vect{AB}$.
 \begin{center}
 \begin{tikzpicture}[general, scale=0.8]
 \draw [quadrillage] (0,0) grid (3,3);
\draw[fill=D1] (1,0) -- (3,1) -- (0,3) -- cycle;
\draw (3,1) node[above] {$A$};
\draw (1,0) node[left] {$B$};
\draw (0,3)  node[left] {$C$};
\end{tikzpicture}
\end{center}
\begin{Correction}%link=Translation d'un triangle
 \begin{center}
 \begin{tikzpicture}[general, scale=0.8]
 \draw [quadrillage] (-2,-1) grid (3,3);
\draw[fill=D1] (1,0) -- (3,1) -- (0,3) -- cycle;
\draw[epais, ->, color=C1] (3,1)-- node[below] {{\boldmath $\Vect{ AB}$}}(1,0);
\draw (3,1) node[above] {$A$};
\draw (0,3)  node[left] {$C$};
\draw[fill=D2] (-1,-1) -- (1,0) -- (-2,2) -- cycle;
\draw (1,0) node[above] {$A'$};
\draw (-1,-1) node[right] {$B'$};
\draw (1,0) node[left] {$B$};
\draw (-2,2)  node[left] {$C'$};
\end{tikzpicture}
\end{center}
\end{Correction}
\end{Exercice}

 \begin{Exercice}[Translation d'une carré]%link=Vecteurs#Vecteurs,type=application,seance=Vecteur associé à une translation
  Construire un carré $ABCD$ de côté 5 cm et  de centre $O$.  Construire l'image de ce carré:
  \begin{enumerate}
   \item dans la translation de vecteur $\Vect{AB}$;
   \item dans la translation de vecteur $\Vect{AC}$;
   \item dans la translation de vecteur $\Vect{OD}$.
  \end{enumerate}
\begin{Correction}%link=Translation d'un carré
 \begin{center}
 \begin{tikzpicture}[general, scale=0.8]
 \draw [quadrillage] (-3,-1) grid (11,11);
\draw[] (0,0)node[below] {$A$} -- (5,0)node[below] {$B$} -- (5,5)node[above] {$C$} -- (0,5)node[above] {$D$}-- cycle;
\draw[epais, ->, color=C1] (0,0)-- node[below] {{\boldmath $\Vect{ AB}$}}(5,0);
\draw[epais,color=C1] (5,0)-- (10,0) -- (10,5) -- (5,5)-- cycle;
\draw[epais, ->, color=D1] (0,0)-- node[below] {{\boldmath $\Vect{ AC}$}}(5,5);
\draw[epais,color=D1] (5,5)-- (10,5) -- (10,10) -- (5,10)-- cycle;
\draw[epais, ->, color=A1] (2.5,2.5)-- node[below] {{\boldmath $\Vect{ OD}$}}(0,5);
\draw[epais,color=A1] (-2.5,2.5)-- (2.5,2.5) -- (2.5,7.5) -- (-2.5,7.5)-- cycle;
\end{tikzpicture}
\end{center}
\end{Correction}
 \end{Exercice}
 
 \begin{Exercice}[Représentants d'un vecteur]%seance=Égalité entre vecteurs%%%%ex 12
Reproduire la figure ci-dessous et construire deux représentants du vecteur de la translation qui transforme $A$ en $B$.
  \begin{center}
   \begin{tikzpicture}[general, scale=0.8]
\draw [quadrillage] (-1,-1) grid (4,4);
\pointGraphique{0}{0}{A}{left}
\pointGraphique{2}{3}{B}{above}
   \end{tikzpicture}
  \end{center}
\begin{Correction}%link=Translation d'un carré
Quelques représentants du vecteur de la translation qui transforme $A$ en $B$
  \begin{center}
   \begin{tikzpicture}[general, scale=0.8]
\draw [quadrillage] (-1,-1) grid (4,4);
\pointGraphique{0}{0}{A}{left}
\pointGraphique{2}{3}{B}{above}
\draw[epais, ->, color=red] (0,0)--(2,3);
\draw[epais, ->, color=red] (-1,0)--(1,3);
\draw[epais, ->, color=red] (2,1)--(4,4);
\draw[epais, ->, color=red] (1,0.5)--(3,3.5);
\draw[epais, ->, color=red] (1,-1)--(3,2);
   \end{tikzpicture}
  \end{center}
\end{Correction}  
   
 \end{Exercice}
%  
% \begin{Exercice}%%%%ex 13
%   Reproduire la figure ci-dessous et construire deux représentants du vecteur de la translation qui transforme $A$ en $B$.  
%  \begin{center}
%   \begin{tikzpicture}[general, scale=0.8]
%\quadrillageSeyes{(0,0)}{(4.8,3.2)}
%\draw (0.8,2.4)--(2.4,0.8)--(4,1.6)--cycle;
%\pointGraphique{0.8}{2.4}{A}{left}
%\pointGraphique{2.4}{0.8}{B}{below}
%\pointGraphique{4}{1.6}{C}{above}
%   \end{tikzpicture}
%  \end{center}  
% \end{Exercice} 
% \begin{Exercice}%%%%ex 14
%  Reproduire la figure ci-dessous et construire trois représentants du vecteur de la translation qui transforme $A$ en $B$.  
%  \begin{center}
%   \begin{tikzpicture}[general, scale=0.8]
%\quadrillageSeyes{(0,0)}{(4.8,4)}
%\draw (1.6,0.8)--(3.2,1.6)--(3.2,3.2)--(1.6,2.4)--cycle;
%\pointGraphique{1.6}{0.8}{A}{left}
%\pointGraphique{3.2}{1.6}{B}{below}
%\pointGraphique{3.2}{3.2}{C}{right}
%\pointGraphique{1.6}{2.4}{N}{left}
%   \end{tikzpicture}
%  \end{center}  
% \end{Exercice} 
% 
% %%%%ex 15
% \begin{Exercice}
%  Construire un triangle $ABC$ rectangle en $A$.\\  
%  Construire le représentant d'origine $A$ du vecteur $\Vect{BC}$.
%  \begin{Correction}
%  ~\\
%  \begin{tikzpicture}[general, scale=0.5]
%\draw (0,0) -- (2,0) -- (0,3) -- cycle;
%\draw (0,0) node[left] {$A$};
%\draw (2,0) node[right] {$B$};
%\draw (0,3)  node[left] {$C$};
%\draw[epais, ->, color=C1] (0,0)--(-2,3);
%\end{tikzpicture}
%  \end{Correction}
% \end{Exercice}
%  
% \begin{Exercice}%%%%ex 16
%   Construire un losange $ABCD$.    \\
%   Construire le représentant d'extrémité $C$ de $\Vect{BD}$.
%  \end{Exercice}
%  
% \begin{Exercice}%%%%ex 17
% \label{2G3_E_1}
%   \`A partir de la figure ci-dessous, citer un vecteur:
%   \begin{enumerate}
%    \item opposé à $\Vect{CD}$;
%    \item de même direction et de même sens que $\Vect{AC}$;
%    \item de même direction  que $\Vect{BC}$ mais de sens contraire;
%    \item égal au vecteur $\Vect{BA}$.
%   \end{enumerate}
%   \begin{center}
%   \begin{tikzpicture}[general]
%\quadrillageSeyes{(0,0)}{(8,7.2)}
%
%\pointGraphique{2.4}{3.2}{A}{left}
%\pointGraphique{0.8}{0.8}{B}{below}
%\pointGraphique{2.4}{0.8}{C}{left}
%\pointGraphique{3.2}{1.6}{D}{below}
%    \draw[epais, ->, color=C1] (0,2.4)--(0.8,2.4);
%    \draw[color=C1] (0.4,2.4)  node[above] {{\boldmath $\Vec u$}};
%    \draw[epais, ->, color=C1] (0.8,4)--(2.4,6.4);
%    \draw[color=C1] (1.6,5)  node[above left] {{\boldmath $\Vec v$}};
%    \draw[epais, ->, color=C1] (3.2,4)--(3.2,5.6);
%    \draw[color=C1] (3.2,4.8)  node[left] {{\boldmath $\Vec w$}};
%    \draw[epais, ->, color=C1] (4,4.8)--(4.8,5.6);
%    \draw[color=C1] (4.4,5.2)  node[above] {{\boldmath $\Vec r$}};
%    \draw[epais, ->, color=C1] (6.4,6.4)--(4.8,2.4);
%    \draw[color=C1] (5.6,3.6)  node[above] {{\boldmath $\Vec s$}};
%    \draw[epais, ->, color=C1] (6.4,6.4)--(5.6,6.4);
%    \draw[color=C1] (6,6.4)  node[above] {{\boldmath $\Vec t$}};
%    \draw[epais, ->, color=C1] (7.2,1.6)--(6.4,0.8);
%    \draw[color=C1] (6.8,1.2)  node[above] {{\boldmath $\Vec p$}};
%    \draw[epais, ->, color=C1] (7.2,4)--(7.2,3.2);
%    \draw[color=C1] (7.2,3.6)  node[left] {{\boldmath $\Vec m$}};
%   \end{tikzpicture}
%   \end{center}
%  \end{Exercice}
%
% \begin{Exercice}%%%%ex 18
%Placer deux points $A$ et $B$ et tracer le vecteur $\Vect{AB}$.   
%   \begin{enumerate}
%    \item Construire un vecteur opposé à $\Vect{AB}$.
%    \item Construire un vecteur de même direction et de même sens que $\Vect{AB}$ et qui n'est pas égal à $\Vect{AB}$.
%    \item Construire un vecteur de même direction  que $\Vect{AB}$ mais de sens contraire et qui n'est pas égal à $\Vect{BA}$.
%   \end{enumerate}
%\end{Exercice}
%
%
%
%\section{Opérations sur les vecteurs}
%
% \begin{Exercice}%%%%ex 20
%Un deltaplane se déplace suivant la translation de vecteur $\Vect{u}$ puis celle de vecteur $\Vec v$.
%\begin{enumerate}
%\item Reproduire la figure ci-dessous.
%\item  Construire l'image du nouveau deltaplane dans sa position finale.
%\end{enumerate}
%   \begin{tikzpicture}[general]
%\quadrillageSeyes{(0,0)}{(8,2.4)}
%\draw[epais] (1.2,0.8)--(1,0)--(2.4,0.8)--(1.6,1.2);
%\draw[color=C1, fill=C2] (0,0)--(2.4,1.6)--(0,2.4)--cycle;
%\draw[color=red, ->, epais] (4,0.8)--(5.6,1.6);
%\draw[color=red] (4.8,1.2) node[above left] {{\boldmath $\Vec u$}};
%\draw[color=G1, ->, epais] (5.6,1.6)--(8,0.8);
%\draw[color=G1] (6.8,1.2) node[above right] {{\boldmath $\Vec v$}};
%   \end{tikzpicture}
%   \end{Exercice}
%   
% \begin{Exercice}\label{2G3_E_construire_somme}%%%%ex 21
% \begin{enumerate}
% \item Reproduire la figure ci-dessous.
% \item Construire les vecteurs suivants.
%\begin{enumerate}
% \item $\Vec w+\Vec r$
% \item $\Vec r+\Vec v$
% \item $\Vec v+\Vec w$
%\end{enumerate}
% \end{enumerate}
% 
%   \begin{tikzpicture}[general]
%\quadrillageSeyes{(0,0)}{(8,3.2)}
%\draw[color=C1, ->, epais] (4,1.6)--(4.8,3.2);
%\draw[color=C1] (4.4,2.4) node[above left] {{\boldmath $\Vec v$}};
%\draw[color=C1, ->, epais] (4,1.6)--(2.4,0);
%\draw[color=C1] (3.6,0.8) node[below left] {{\boldmath $\Vec w$}};
% \draw[color=C1, ->, epais] (4,1.6)--(5.6,1.6);
% \draw[color=C1] (4.8,1.6) node[above left] {{\boldmath $\Vec r$}};
%\end{tikzpicture}
%\begin{Correction}
%~\\
% \begin{tikzpicture}[general]
%\quadrillageSeyes{(2,0)}{(7,3.2)}
%\draw[color=C1, ->, epais] (4,1.6)--(4.8,3.2);
%\draw[color=C1] (4.4,2.4) node[above left] {{\boldmath $\Vec v$}};
%\draw[color=C1, ->, epais] (4,1.6)--(2.4,0);
%\draw[color=C1] (3.6,0.8) node[below left] {{\boldmath $\Vec w$}};
% \draw[color=C1, ->, epais] (4,1.6)--(5.6,1.6);
% \draw[color=C1] (5.1,1.6) node[above left] {{\boldmath $\Vec r$}};
% \draw[color=C1, ->, epais] (4,1.6)--(5.6,1.6);
% \draw[color=A1, ->, epais] (4,1.6)--(4,0);
% \draw[color=A1] (5.2,0.4) node[left] {{\boldmath $\Vec w+\Vec r$}};
% \draw[color=A1, ->, epais] (4,1.6)--(6.4,3.2);
% \draw[color=A1] (6.8,2.6) node[left] {{\boldmath $\Vec r+\Vec v$}};
% \draw[color=A1, ->, epais] (4,1.6)--(3.2,1.6);
% \draw[color=A1] (3.8,1.9) node[left] {{\boldmath $\Vec v+\Vec w$}};
%\end{tikzpicture}
%
%\end{Correction}
%
%   \end{Exercice}
%   
% \begin{Exercice}%%%%ex 22
% \begin{enumerate}
% \item Reproduire la figure ci-dessous.
% \item Construire les vecteurs suivants.
%\begin{enumerate}
% \item $\Vect{BC}+\Vect{CD}$
% \item $\Vect{BA}+\Vect{BC}$
%\end{enumerate}
% \end{enumerate}
% 
%   \begin{tikzpicture}[general]
%\quadrillageSeyes{(0,0)}{(8,3.2)}
%\foreach \x/\y/\N/\pos in {1.6/0/A/above,3.2/1.6/B/left,6.4/0.8/C/below,6.4/2.4/D/above}
%{\pointGraphique{\x}{\y}{\N}{\pos}}
%\end{tikzpicture}
%   \end{Exercice}
%   
% \begin{Exercice}%%%%ex 23
% \begin{enumerate}
% \item Reproduire la figure ci-dessous.
% \item Construire les vecteurs suivants.
%\begin{enumerate}
% \item $\Vect{AB}+\Vect{CD}$
% \item $\Vect{BA}+\Vect{EF}$
% \item $\Vect{CD}+\Vect{FE}$
%\end{enumerate}
% \end{enumerate}
% 
%   \begin{tikzpicture}[general]
%\quadrillageSeyes{(0,0)}{(8,3.2)}
%\foreach \x/\y/\N/\pos in {1.6/1.6/A/left,2.4/0.4/B/left,3.2/1/C/below,4.8/2.4/D/above,4.8/1.6/E/left,6.4/1.6/F/below}
%{\pointGraphique{\x}{\y}{\N}{\pos}}
%\end{tikzpicture}
%\end{Exercice}
%
%
%
% \begin{Exercice}%%%%ex 24
%Reproduire la figure ci-dessous et construire un représentant des vecteurs suivants.
%\begin{enumerate}
% \item $\Vec u+\Vect{AB}$
% \item $\Vec v+\Vect{CB}$
% \item $\Vec w+\Vect{DE}$
%\end{enumerate}
%   
%\begin{tikzpicture}[general]
%\quadrillageSeyes{(0,0)}{(8,4)}
%\foreach \x/\y/\N/\pos in {0.8/1.6/A/below,4/0.8/B/left,4.8/2.4/C/below,6.4/3.2/D/below right,5.6/0.8/E/left}
%{\pointGraphique{\x}{\y}{\N}{\pos}}
%\foreach \a/\b/\c/\d/\N/\pos in {0.8/1.6/1.6/2.4/u/left, 3.2/3.2/4.8/2.4/v/above,7.2/2.4/6.4/0.8/w/right}
%{\draw[epais, ->, color=A1] (\a,\b)--(\c,\d);
%\draw[color=A1] (0.5*\a+0.5*\c, 0.5*\b+0.5*\d) node[\pos] {{\boldmath $\Vect{\N}$}};
%}
%\end{tikzpicture}
%\end{Exercice}
%
% \begin{Exercice}%%%%ex 25
%    \label{2G3_base_motif} Compléter les égalités en n'utilisant que les points de la figure ci-dessous.
% \begin{enumerate}
%  \item $\Vect{IB}=\Vect{\ldots A}+\Vect{A \ldots }$
%  \item $\Vect{HG}+\Vect{\ldots\phantom{0.5}}=\Vect{HF}$
%  \item $\Vect{D\ldots}+\Vect{C\ldots}=\Vect{ \ldots B}$
%    \item $\Vect{E \ldots}+\Vect{\ldots E}=\Vect{\ldots\phantom{0.5}} $
% \end{enumerate}
% \begin{enumerate}
% \item $\Vect{A\ldots}=\Vect{A\ldots}+\Vect{B\ldots}+\Vect{CM}$
% \item $\Vect{FE}+\Vect{\ldots\phantom{0.5}}=\Vect{0}$
% \end{enumerate}   
%
%    \begin{center}   
%    \begin{tikzpicture}[general ,scale=0.6]
%    \draw[quadrillage] (-1,-1) grid (7,6);
%  \draw[color=C1, epais](0,2)--(1,0)--(2,0)--(3,1)--(4,0)--(5,0)--(4,2)--(6,4)--(4,4)--(3,5)--cycle;  
%  \pointGraphique{3}{5}{A}{above left}
%  \pointGraphique{4}{4}{B}{above right}
%  \pointGraphique{6}{4}{C}{above right}
%  \pointGraphique{4}{2}{D}{above left}
%  \pointGraphique{5}{0}{E}{below right}
%  \pointGraphique{4}{0}{F}{below left}
%  \pointGraphique{3}{1}{G}{above left}
%  \pointGraphique{2}{0}{H}{below right}
%  \pointGraphique{1}{0}{I}{below left}
%  \pointGraphique{0}{2}{J}{above left}
% \end{tikzpicture}
% \end{center}
%
% 
% \end{Exercice}
% 
% \begin{Exercice}%%%%ex 26
%  \'Ecrire le plus simplement possible. 
%  
%  \begin{enumerate}
%   \item $\Vect{BD}+\Vect{DA}$
%   \item $\Vect{BD}+\Vect{AA}$
%   \item $\Vect{BD}+\Vect{DB}$
%   \item $\Vect{BD}-\Vect{BA}$
%   \item $\Vect{BD}+\Vect{AD}+\Vect{BA}$
%   \item $\Vect{BD}-\Vect{BA}+\Vect{DA}-\Vect{DB}$
%  \end{enumerate}
%
% \end{Exercice}
%
% \begin{Exercice}%%%%ex 27
%  \'Ecrire le plus simplement possible. 
%  \begin{enumerate}
%   \item $\Vect{MB}-\Vect{MD}$
%   \item $\Vect{CB}-\Vect{CD}-\Vect{BD}$
%   \item $\Vect{BD}-\Vect{BA}+\Vect{MA}-\Vect{MD}$
%   \item $\Vect{BD}-\Vect{MC}-\Vect{BM}+\Vect{DB}$
%   \item $\Vect{MA}+\Vect{EM}-\Vect{CA}-\Vect{EC}$
%   \item $-\Vect{AU}+\Vect{SH}-\Vect{ST}+\Vect{MU}$
%  \end{enumerate}
% \end{Exercice}
%
% \begin{Exercice}%%%%ex 28
% En utilisant le motif de l'Exercice \ref{2G3_base_motif}:\\ donner un vecteur égal à:
% 
% \begin{enumerate}
%  \item $\Vect{DE}+\Vect{HI}$
%  \item $\Vect{GF}+\Vect{CB}$
%  \item $\Vect{AJ}+\Vect{IE}$
%  \item $\Vect{BG}+\Vect{GH}$
%   \item $\Vect{BC}+\Vect{CB}+\Vect{BC}$
%  \item $\Vect{IJ}+\Vect{CF}+\Vect{JC}+\Vect{FE}$
%  \item $\Vect{AB}+\Vect{EF}+\Vect{DE}$  
%  \item $\Vect{HF}+\Vect{ED}+\Vect{CD}$
% \end{enumerate}
% \end{Exercice}
%
%
% \begin{Exercice} [Vecteurs et représentants]%%%%ex 29
% \begin{center}
%        \begin{tikzpicture}[general, scale=0.8]
%            \quadrillageSeyes{(-1.6,-0.8)}{(3.2,2.4)} 
%            \pointGraphique{0.8}{1.6}{A}{above}
%            \pointGraphique{2.4}{0.8}{B}{right}
%            \pointGraphique{0}{0}{C}{left}
%        \end{tikzpicture}
%        \end{center}
%
%    \begin{enumerate}
%        \item Reproduire la figure ci-dessus.
%        \item Placer les points $E$ et $F$ tels que:
%        \begin{itemize}
%        	\item  $\Vect{CE}=\Vect{BA}$
%        	\item  $\Vect{FB}=\Vect{BC}$
%		\end{itemize}   	
%        \item Déterminer le représentant:
%        	\begin{enumerate}
%        		\item du vecteur $\Vect{BC}$ d'origine $A$;
%        		\item du vecteur $\Vect{BA}$ d'extrémité $C$.
%        	\end{enumerate}
%        \item Représenter les vecteurs $\Vect{u}$ et $\Vect{v}$ tels que : 
%           \begin{itemize}
%            \item     $\Vect{u}=\Vect{BC}+\Vect{AC}$
%            \item     $\Vect{v}=\Vect{AB}-\Vect{AC}$
%           \end{itemize}
%        \item Quelle est la nature du quadrilatère $AEBF$? Justifier.
%    \end{enumerate}
%\end{Exercice}
%%30
%\begin{Exercice}[Petites démonstrations]
%Soit $A$, $B$ et $C$ trois points.
%\begin{enumerate}
%\item Construire le point $D$ tel que $\Vect{AB}=\Vect{CD}$.
%\item Construire le point $E$ tel que $\Vect{AB}=\Vect{EC}$.
%\item Que peut-on dire du point $E$? Justifier.
%\end{enumerate}
%\end{Exercice}
%
% \begin{Exercice}[Dans un parallélogramme...]%%%%ex 31
%    $EFGH$ est un parallélogramme de centre $O$.
%    \begin{enumerate}
%        \item Construire les points $S$ et $T$ vérifiant:
%        \begin{itemize}
%        	\item $\Vect{OT}$ = $\Vect{OE}$ + $\Vect{OF}$
%        	\item $\Vect{OS}$ = $\Vect{OG}$ + $\Vect{OH}$
%        \end{itemize}
%        \item Démontrer
%            que $\Vect{OT}$ + $\Vect{OS}$ =
%            $\Vect{0}$.\\ Que peut-on en
%            déduire?
%            \end{enumerate}
%            
%\end{Exercice}
%
% \begin{Exercice} [Dans un triangle rectangle...]%%%%ex 32
%    Soit $ABC$ un triangle rectangle en $A$.
%    \begin{enumerate}
%        \item Construire les points $D$ et $E$ tels que:
%        	\begin{itemize}
%        		\item $\Vect{AD}=\Vect{BA}$
%        		\item $\Vect{CE}=\Vect{CB}+\Vect{CD}$
%        	\end{itemize}
%        \item Quelle est la nature du quadrilatère $BCDE$ ? Justifier.
%    \end{enumerate}
% \end{Exercice}
%    
% \begin{Exercice}[Démonstration]%%%%ex 33
%    Tous les résultats devront être démontrés.
%    \begin{enumerate}
%        \item Construire un parallélogramme $ABCD$ de centre $O$. \\ Nommer
%$I$ le milieu de $[OC]$.
%        \item Construire $A'$ le symétrique de $A$ par rapport à $D$ \\et
%$O'$ le symétrique de $O$ par rapport à $B$.
%        \item 
%            \begin{enumerate}
%                \item Démontrer que $\Vect{A'C}=\Vect{DB}$.
%                \item Démontrer que $\Vect{DB}=\Vect{OO'}$.
%                \item En déduire que $I$ est le milieu de $[A'O']$.
%			 \end{enumerate}
%	 \end{enumerate}
% 
% \end{Exercice}
%
%
% %%%% ex33
%
% \begin{Exercice}%%%%ex 34
% \label{2G3_E_ex34}
%Reproduire la figure ci-dessous et construire les vecteurs suivants.
%\begin{enumerate}
% \item $\Vect{u_1}=\Vec w-\Vec r$
% \item $\Vect{u_2}=\Vec r-\Vec v$
% \item $\Vect{u_3}=\Vec v-\Vec w$
% \item $\Vect{u_4}=\Vec r-\Vec w$
% \item $\Vect{u_5}=\Vec v-\Vec r$
% \item $\Vect{u_6}=\Vec w-\Vec r$
%\end{enumerate}
%\begin{enumerate}
%\setcounter{enumi}{6}
%\item Quelles remarques peut-on faire?
%\end{enumerate}
%\begin{center}
%   \begin{tikzpicture}[general, scale=0.8]
%\quadrillageSeyes{(0,0)}{(8,3.2)}
%\draw[color=C1, ->, epais] (4,1.6)--(4,3.2);
%\draw[color=C1] (4,2.4) node[left] {{\boldmath $\Vec v$}};
%\draw[color=C1, ->, epais] (4,1.6)--(1.6,0.8);
%\draw[color=C1] (3.2,0.8) node[below] {{\boldmath $\Vec w$}};
% \draw[color=C1, ->, epais] (4,1.6)--(7.2,0.8);
% \draw[color=C1] (5.6,1.2) node[above] {{\boldmath $\Vec r$}};
%\end{tikzpicture}
%\end{center}
%   \end{Exercice}
%   
% \begin{Exercice}%%%%ex 35
%Même consigne qu'à l'Exercice \ref{2G3_E_ex34}.
%\begin{enumerate}
% \item $\Vect{BC}-\Vect{CD}$
% \item $\Vect{BA}-\Vect{BC}$
% \item $\Vect{DA}-\Vect{AB}$
%\end{enumerate} 
%\begin{center}
%   \begin{tikzpicture}[general, scale=0.8]
%\quadrillageSeyes{(0,0)}{(8,3.2)}
%\foreach \x/\y/\N/\pos in {5.6/0.8/A/below,7.2/2.4/B/left,1.6/1.6/C/below,3.2/0.8/D/below}
%{\pointGraphique{\x}{\y}{\N}{\pos}}
%\end{tikzpicture}
%\end{center}
%\end{Exercice}
%   
% \begin{Exercice}%%%%ex 36
%  Même consigne qu'à l'Exercice \ref{2G3_E_ex34}.
%\begin{enumerate}
% \item $\Vect{AB}-\Vect{CD}$
% \item $\Vect{BA}-\Vect{EF}$
% \item $\Vect{CD}-\Vect{FE}$
%\end{enumerate}
%\begin{enumerate}
%\setcounter{enumi}{3}
%\item Que dire du quadrilatère $CFEO$?
%\end{enumerate}
%\begin{center}
%\begin{tikzpicture}[general, scale=0.8]
%\quadrillageSeyes{(0,0)}{(8,3.2)}
%\foreach \x/\y/\N/\pos in {7.2/0.8/A/below left,4/1.6/B/below left,0.8/0.8/C/below left,4.8/0.8/D/below right,6.4/2.4/E/above right,2.4/2.4/F/above left}
%{\pointGraphique{\x}{\y}{\N}{\pos}}
%\end{tikzpicture}
%\end{center}
%\end{Exercice}
%
% \begin{Exercice}%%%%ex 37
%Reproduire la figure ci-dessous.\\ On considère les vecteurs suivants.
%\begin{itemize}
% \item $\Vect{u_1}=\Vec u-\Vect{AB}$
% \item $\Vect{v_1}=\Vec v-\Vect{CD}$
% \item $\Vect{w_1}=\Vec w-\Vect{DE}$
%\end{itemize}
%Construire un représentant de:
%\begin{enumerate}
%\item $\Vect{u_1}$ d'origine $E$;
%\item $\Vect{u_2}$ d'origine $A$;
%\item $\Vect{u_3}$ d'origine $C$;
%\item $\Vect{u_1}$ d'extrémité $C$;
%\item $\Vect{u_2}$ d'extrémité $E$;
%\item $\Vect{u_3}$ d'extrémité $A$.
%\end{enumerate}
%\begin{center}   
%\begin{tikzpicture}[general, scale=0.8]
%\quadrillageSeyes{(0,0)}{(8,4)}
%\foreach \x/\y/\N/\pos in {0.8/4/B/below left,4/0.8/A/below left,4.8/2.4/C/below right,4/1.6/D/below right,5.6/2.4/E/above left}
%{\draw (\x,\y) node {$+$}; 
%\draw (\x,\y) node[\pos] {$\N$};}
%\foreach \a/\b/\c/\d/\N/\pos in {1.6/2.4/0.8/4/u/left, 4/1.6/2.4/1.6/v/above,6.4/2.4/7.2/2.4/w/above}
%{\draw[epais, ->, color=A1] (\a,\b)--(\c,\d);
%\draw[color=A1] (0.5*\a+0.5*\c, 0.5*\b+0.5*\d) node[\pos] {{\boldmath $\Vect{\N}$}};
%}
%\end{tikzpicture}
%\end{center}
%\end{Exercice}
%\section{Coordonnées d'un vecteur}
%
%%%%38
%\begin{Exercice}\label{2G3_E_lire_coord}
%Lire les coordonnées des vecteurs suivants.%dans le repère $(O;I,J)$ ci-dessous.
%
% \begin{enumerate}
%  \item $\Vect{AB}$
%  \item $\Vect{AC}$
%  \item $\Vect{CA}$
%  \item $\Vect{DE}$
%  \item $\Vect{AE}$
%  \item $\Vect{AF}$
% \end{enumerate}
% \begin{center}
%\begin{tikzpicture}[general, scale=0.40]
%\draw [quadrillage] (-6,-3) grid (6,6);
%\axeOI{-6}{6}
%\axeOJ{-3}{6}
%\origineO
%\foreach \N/\x/\y in {A/1/1,B/3/4, C/-4/3, D/-2/-2, E/1/-2, F/3/0}
%{\pointGraphique{\x}{\y}{\N}{above}}\end{tikzpicture}
% \end{center}
% \begin{Correction}
%  ~\\
% %%%%%%%%%%%%% avec les noms des vecteurs, ne rentre pas sur une ligne 
% \begin{enumerate}
% \item  $\left(\begin{array}{c}2\\3\end{array}\right)$
%  \item  $\left(\begin{array}{c}-5\\2\end{array}\right)$
%  \item  $\left(\begin{array}{c}5\\-2\end{array}\right)$
%  \item  $\left(\begin{array}{c}3\\0\end{array}\right)$
%  \item $\left(\begin{array}{c}0\\-3\end{array}\right)$
%  \item  $\left(\begin{array}{c}2\\-1\end{array}\right)$
% \end{enumerate}
% \end{Correction}
%\end{Exercice}
%
%\begin{Exercice}%%%39
%Dans le repère $(O;I,J)$ ci-dessous,
%\begin{enumerate}
%\item lire les coordonnées des points;
%\item calculer les coordonnées des vecteurs suivants. 
% \begin{itemize}
%  \item $\Vect{AB}$
%  \item $\Vect{AC}$
%  \item $\Vect{BJ}$
%  \item $\Vect{BD}$
%  \item $\Vect{FA}$
%  \item $\Vect{FJ}$
%  \item $\Vect{GF}$
%  \item $\Vect{BG}$
% \end{itemize}
% \item  Dans cette liste, quels vecteurs sont égaux? \\ Lesquels sont opposés? 
% \end{enumerate}
%  \begin{center}
%\begin{tikzpicture}[general, scale=0.40]
%\draw [quadrillage] (-6,-3) grid (6,9);
%\axeOI{-6}{6}
%\axeOJ{-3}{9}
%\foreach \N/\x/\y/\pos in {A/-4/2/left, B/1/4/right, C/-5/-1/left, D/2/8/left, F/-1/-2/left, G/4/0/below}
%{\pointGraphique{\x}{\y}{\N}{\pos}}
%\end{tikzpicture}
% \end{center}
%\end{Exercice}
%
%\begin{Exercice}%%%40
%Lire les coordonnées des vecteurs suivants dans le repère $(O;I,J)$ ci-dessous.
%
% \begin{enumerate}
%  \item $\Vect{AB}$
%  \item $\Vect{AD}$
%  \item $\Vect{CA}$
%  \item $\Vect{DE}$
%  \item $\Vect{AE}$
%  \item $\Vect{CF}$
% \end{enumerate}
% \begin{center}
%\begin{tikzpicture}[general, scale=0.7]
%\clip (1,0)rectangle(11,5.22);
%\foreach \x in {0, 1, 2, 3, 4, 5, 6, 7, 8, 9, 10, 11, 12}
%{\foreach \y in {0, 1.74, 3.48}
%{\draw[quadrillage] (0+\x,0+\y)--(1+\x,0+\y)--(0.5+\x,0.87+\y)--cycle;
%\draw[quadrillage] (0.5+\x,0.87+\y)--(1.5+\x,0.87+\y)--(1+\x,1.74+\y)--cycle;
%}}
%\draw[quadrillage] (0,5.22)--(11,5.22);
%\draw[axe] (0,1.74) -- (11,1.74);
% \draw[axe] (3,0) -- (6,5.22);
% \draw[color=black] (3.9,1.74) node[below left] {{ $O$}};
% \draw[color=black] (5,1.74) node {{\boldmath $+$}};
% \draw[color=black] (5,1.74) node[below] {{$I$}};
% \draw[color=black] (4.5,2.61) node {{\boldmath $-$}};
% \draw[color=black] (4.5,2.61) node[left] {{$J$}};
% \foreach \N/\x/\y in {A/6.5/2.61,B/9.5/4.35, C/2/3.48, D/6.5/.87, E/9.5/0.87, F/1.5/2.61}
% {\pointGraphique{\x}{\y}{\N}{above}}
%\end{tikzpicture}
% \end{center}
%\end{Exercice}
%\begin{Exercice}%%%41
%Construire un repère $(O;I,J)$ orthogonal.
%\begin{enumerate}
%\item Placer le point $A(-3;4)$.
%\item Construire un représentant du vecteur $\Vec u$ \\de coordonnées $\left(\begin{array}{c}4\\-3\end{array}\right)$.
%\item Placer les points $B$ et $C$ tels que:
%\begin{itemize}
%\item $\Vect{AB}=\Vec u$
%\item $\Vect{CA}=\Vec u$
%\end{itemize}
%\item Calculer les coordonnées des points $B$ et $C$. 
%\item Que peut-on dire du point $A$? Justifier. 
%\end{enumerate}
%\end{Exercice}
%
%%%%42
%\begin{Exercice}\label{2G3_E_construire_coord_vecteur}Construire un repère $\left(O; I, J\right)$ et tracer deux représentants du vecteur $\Vec u$ $\left(\begin{array}{c}-2\\1\end{array}\right)$,\\ l'un d'origine $I$ et l'autre d'extrémité $J$.
%\begin{Correction}
% ~\\
%\begin{center}
%\begin{tikzpicture}[general, scale=0.60]
%\draw [quadrillage] (-1,-1) grid (2,2);
%\axeOI{-1}{2}
%\axeOJ{-1}{2}
%\origineO
%\draw[color=A1, ->, epais] (1,0)--(-1,1);
%\draw[color=C1, ->, epais] (2,0)--(0,1);
%\end{tikzpicture}
%\end{center}
%\end{Correction}
%\end{Exercice}
%
%\begin{Exercice}[Coordonnées de vecteurs]%%%43
% Dans le plan muni d'un repère, on considère les points $A(1;2)$, $B(-2; 5)$ et $C(-3;-3)$. 
% 
% Calculer les coordonnées des vecteurs $\Vect{AB}$, $\Vect{CA}$ et $\Vect{BC}$.
%\end{Exercice}
%
%%%%44
%\begin{Exercice}\label{2G3_E_reperer_point_egalite}
% Dans le plan muni d'un repère, on considère les points $E(2;-1)$, $F(-3;4)$ et $G(1;4)$. 
% 
% Déterminer les coordonnées du point $H$ pour que $EFGH$ soit un parallélogramme.
% \begin{Correction}
% $H(-1;6)$
% \end{Correction}
%\end{Exercice}
%
%\begin{Exercice}[Parallélogramme]%%%45
%Dans un plan muni d'un repère, on considère les points $A(3;5)$, $B(2;-1)$, $C(-2;-4)$ et $D(-1;2)$. 
% 
%Prouver que $ABCD$ est un parallélogramme. 
%\end{Exercice}
%
%\begin{Exercice}%%%46
%Construire un repère $(O; I, J)$ orthogonal.
%\begin{enumerate}
%\item  Placer les points $A(3;-9)$  et $B(-1;-5)$. 
%\item Placer les points $C$ et $D$ tels que le quadrilatère $ABCD$ soit un parallélogramme de centre $I$.
% \item Déterminer les coordonnées des vecteurs suivants.  
% \begin{itemize}
%   \item $\Vect{AB}$ 
%  \item $\Vect{DC}$
%  \item $\Vect{AD}$
% \end{itemize}
%\end{enumerate}
%\end{Exercice}
%
%\begin{Exercice}%%%47
% Dans le plan muni d'un repère, les coordonnées des points $A$ et $B$ sont respectivement $(5;-6)$ et $(-2;6)$. Le point $A$ est le milieu de $[BC]$. 
% 
% Déterminer les coordonnées des vecteurs $\Vect{AB}$ et $\Vect{CA}$.
%\end{Exercice}
%
%\begin{Exercice}%%%48
% Dans le plan muni d'un repère orthonormal, on considère les points $A$, $B$ et $C$ respectivement de coordonnées $(1;4)$, $(4;6)$ et $(2;3)$. 
% \begin{enumerate}
%  \item Quelles sont les coordonnées du point $D$ tel que $ABCD$ soit un parallélogramme?
%  \item Prouver que $ABCD$ est aussi un losange. 
% \end{enumerate}
%\end{Exercice}
%
%
%\begin{Exercice}%%%49
% Dans le plan muni d'un repère orthonormal, les coordonnées des points $A$, $B$ et $D$ sont respectivement $(-2;5)$, $(0;9)$ et $(8;0)$. 
% \begin{enumerate}
%  \item Quelles sont les coordonnées du point $C$ tel que $ABCD$ soit un parallélogramme?
%  \item Prouver que $ABCD$ est aussi un rectangle.
% \end{enumerate}
%\end{Exercice}
%
%%%%%%%%%%%%%%%%50
%\begin{Exercice}\label{2G3_E_repere_point_somme}Le plan est muni d'un repère $(O;I,J)$. %%%50
%\begin{enumerate}
% \item Lire les coordonnées des vecteurs $\Vec u$, $\Vec v$ et $\Vec w$.
% \item Calculer les coordonnées des vecteurs suivants.
% \begin{enumerate}
%  \item $\Vec u + \Vec v$
%\item $\Vec u - \Vec v$
%\item $\Vec u + \Vec w$
%\item $\Vec u - \Vec w$
% \end{enumerate}
%\end{enumerate}
% \begin{center}
%\begin{tikzpicture}[general, scale=0.5]
%\draw [quadrillage] (-5,-5) grid (5,5);
%\axeOI{-5}{5}
%\axeOJ{-5}{5}
%\origineO
%\foreach \N/\a/\b/\c/\d/\p/\e in {u/-1.5/0.5/-0.5/2.5/left/A1, v/0.5/-1/2/-2/above/G1, w/-2/-2/-1/0/below right/F1}
%{\draw (\a+\c,\b+\d) 
%node[\p, color=\e] {{\boldmath $\Vect{\N}$}};
%\draw[->, color=\e,epais] (2*\a,2*\b)--(2*\c,2*\d);
%}\end{tikzpicture}
% \end{center}
% \begin{Correction}
% ~\\
% \begin{enumerate}
% \item $\Vec u$ $\left(\begin{array}{c}2\\4\end{array}\right)$, $\Vec v$ $\left(\begin{array}{c}3\\-2\end{array}\right)$, \\$\Vec w$ $\left(\begin{array}{c} 2\\4\end{array}\right)$
% \item \begin{enumerate}
% \item $\left(\begin{array}{c}5\\2\end{array}\right)$
% \item  $\left(\begin{array}{c}-1\\6\end{array}\right)$
%\item   $\left(\begin{array}{c}4\\8\end{array}\right)$
%\item  $\left(\begin{array}{c}0\\0\end{array}\right)$
% \end{enumerate}
% \end{enumerate}
%\end{Correction}
%\end{Exercice}
%
%
%\begin{Exercice}\label{2G4_construiresomme}%%%51
%Reproduire la figure suivante et
% placer le point $B$ tel que $\Vect{AB}=\Vec u + \Vec v$. 
% 
% Lire les coordonnées du vecteur $\Vect{AB}$.
% 
% \begin{center}
%\begin{tikzpicture}[general, scale=0.5]
%\draw [quadrillage] (-5,-2) grid (5,5);
%\axeOI{-5}{5}
%\axeOJ{-2}{5}
%\origineO
%\foreach \N/\x/\y in {A/-4/1}
%{\draw (\x,\y) node[left] {$\N$};
%\draw (\x,\y) node { $+$};}
%\draw[->, color=A1,epais] (-4,1)--(-3,-1);
%\draw[color=A1] (-3,-1) node[below, left] {{\boldmath $\Vec u$}};
%\draw[->, color=C1,epais] (-4,1)--(2,4);
%\draw[color=C1] (-1,3) node[left] {{\boldmath $\Vec v$}};
%\end{tikzpicture}
% \end{center}
%\end{Exercice}
%
%%%%52
%\begin{Exercice} Même consigne que l'Exercice \ref{2G4_construiresomme}.
%
%\begin{center}
%\begin{tikzpicture}[general, scale=0.5]
%\draw [quadrillage] (-5,-4) grid (5,5);
%\axeOI{-5}{5}
%\axeOJ{-4}{5}
%\origineO
%\foreach \N/\x/\y in {A/-2/-3}
%{\draw (\x,\y) node[left] {$\N$};
%\draw (\x,\y) node { $+$};}
%\draw[->, color=A1,epais] (-2,-3)--(-4,-1);
%\draw[color=A1] (-4,-2) node {{\boldmath $\Vec u$}};
%\draw[->, color=C1,epais] (-4,-1)--(4,4);
%\draw[color=C1] (3.5,3) node {{\boldmath $\Vec v$}};
%\end{tikzpicture}\end{center}
%                 
%\end{Exercice}
%
%%%%53
%\begin{Exercice}Même consigne que l'Exercice \ref{2G4_construiresomme}.
%
%\begin{center}
%\begin{tikzpicture}[general, scale=0.5]
%\draw [quadrillage] (-5,-5) grid (5,4);
%\axeOI{-5}{5}
%\axeOJ{-5}{4}
%\origineO
%\foreach \N/\x/\y in {A/-4/-4}
%{\draw (\x,\y) node[below] {$\N$};
%\draw (\x,\y) node { $+$};}
%\draw[->, color=A1,epais] (-4,-4)--(-3,-1);
%\draw[color=A1] (-3.5,-2.5) node[left] {{\boldmath $\Vec u$}};
%\draw[->, color=C1,epais] (5,-4)--(2,3);
%\draw[color=C1] (4,-1.5) node[left] {{\boldmath $\Vec v$}};
%\end{tikzpicture}\end{center}
%                 
%\end{Exercice}
%
%%54
%\begin{Exercice}Même consigne que l'Exercice \ref{2G4_construiresomme}.
%
%\begin{center}
%\begin{tikzpicture}[general, scale=0.5]
%\draw [quadrillage] (-5,-5) grid (5,5);
%\axeOI{-5}{5}
%\axeOJ{-5}{5}
%\origineO
%\foreach \N/\x/\y in {A/-3/4}
%{\draw (\x,\y) node[right] {$\N$};
%\draw (\x,\y) node { $+$};}
%\draw[->, color=A1,epais] (-4,-4)--(2,-4);
%\draw[color=A1] (-1,-4) node[below] {{\boldmath $\Vec u$}};
%\draw[->, color=C1,epais] (4,-3)--(-4,3);
%\draw[color=C1] (-3,1) node {{\boldmath $\Vec v$}};
%\end{tikzpicture}
%\end{center}
%\end{Exercice}
%
%%55
%
%\begin{Exercice} Reproduire la figure ci-dessous.
%\begin{enumerate}
% \item  La compléter avec les points suivants.\begin{itemize}
% \item $D(4,2)$
%  \item $E(1;-2)$
%  \item $F(-3;1)$ \end{itemize} 
%  \item  Placer les points $G$, $H$ et $K$ tels que:
%  \begin{itemize}
%  \item $\Vect{AG}=\Vect{CB}+\Vect{CE}$
%  \item $\Vect{CH}=\Vect{CA}+\Vect{CF}$
%  \item $\Vect{BK}=\Vect{AD}+\Vect{CE}$
%  \end{itemize}
%\item Lire leurs coordonnées.
%\item Les vérifier par le calcul.
%\end{enumerate}
%\begin{center}
%\begin{tikzpicture}[general, scale=0.5]
%\draw [quadrillage] (-5,-3) grid (5,7);
%\axeOI{-5}{5}
%\axeOJ{-3}{7}
%\origineO
%\foreach \x/\y/\P/\N in {-2/2/left/A,3/6/right/B,-1/-2/below/C} 
%{\pointGraphique{\x}{\y}{\N}{\P}}
%\end{tikzpicture}
%\end{center}
%\end{Exercice}
%
%%56
%\begin{Exercice} Reproduire le graphique ci-dessous. Calculer les coordonnées des vecteurs suivants et les représenter.% sur le graphique. 
%
% \begin{enumerate}
%  \item $\Vect{AB}+\Vect{BC}$
%  \item $\Vect{BE}+\Vect{BC}$
%  \item $\Vect{EF}+\Vect{FA}$
%  \item $\Vect{AB}+\Vect{BA}$
%  \item $\Vect{GB}-\Vect{BD}$
%  \item $\Vect{FG}-\Vect{CG}$
% \end{enumerate}
% \begin{center}
%\begin{tikzpicture}[general, scale=1.2]
%\draw [quadrillage, step=0.333] (-2,-1) grid (2,1.667);
%\axeOI{-2}{2}
%\axeOJ{-1}{1.667}
%\origineO
%\foreach \x/\y/\pos/\N in {-1.333/0.667/left/A,0.333/1.333/right/B, -0.333/-0.333/left/C, 0.667/0.667/left/D, 0/0.333/left/E, -0.333/-0.667/left/F, 1.333/0/below/G}
%{\pointGraphique{\x}{\y}{\N}{\pos}}
%\end{tikzpicture}  
% \end{center}
%\end{Exercice}
%
%
%
%%57
%\begin{Exercice}Construire un repère orthogonal.
%\begin{enumerate}
%\item  Placer les points suivants.
%\begin{itemize}
% \item $A(-2;3)$
% \item $B(-1;-2)$ 
% \item $C(3;2)$
% \item $D(4;-2)$
% \item $E(-3;1)$ 
% \item $F(3;-3)$ 
% \item $G(2;3)$
% \item $H(5;1)$
%\end{itemize}
%\item Construire le vecteur $\Vect{AB}$ et un représentant de $-\Vect{AB}$. Lire leurs coordonnées. 
%\item Construire un représentant de chacun des vecteurs suivants et lire leurs coordonnées.
%\begin{enumerate}
%\item $\Vect{AB}-\Vect{CA}$
%\item $\Vect{AB}-\Vect{AC}$
%\item $\Vect{EF}-\Vect{GH}$
%\end{enumerate}
% \end{enumerate}
%\end{Exercice}
%
%%58
%\begin{Exercice}
% Dans le plan muni d'un repère, $\Vec u$ et $\Vec v$ ont pour coordonnées respectives $\left( \begin{array}{c}2\\-3\end{array}\right)$ et $\left( \begin{array}{c}-1\\5\end{array}\right)$. \\
%
%Calculer les coordonnées de $\Vec w$, $\Vec m$ et $\Vec z$ tels que:
%
%\begin{itemize}
% \item $\Vec u +\Vec w=\Vec v$ 
%\item $\Vec u -\Vec m=\Vec v$ 
%\item $\Vec z -\Vec u=\Vec v$
%\end{itemize}
%\end{Exercice}
%
%
%
%\begin{Exercice}Construire un repère orthogonal. %%%59
%\begin{enumerate}
%\item  Placer les points suivants.
%\begin{itemize}
% \item $A(0;0)$
% \item $B(1;-2)$ 
% \item $C(2;-3)$
% \item $D(-5;-4)$
% \item $E(-3;1)$ 
% \item $F(3;-3)$ 
% \item $G(1;5)$
% \item $H(-7;-2)$
%\end{itemize}
%\item Construire un représentant des vecteurs suivants et lire leurs coordonnées.
%\begin{itemize}
% \item Le vecteur qui, ajouté à $\Vect{DC}$, donne $\Vect{DA}$.
% \item Le vecteur qui, ajouté à $\Vect{EF}$, donne $\Vect{EH}$.
% \item Le vecteur qui, ajouté à $\Vect{BC}$, donne $\Vect{GH}$.
% \item Le vecteur qui, ajouté à $-\Vect{AB}$, donne $\Vect{EF}$.
%\end{itemize}
%\end{enumerate}
%\end{Exercice}
%
%%60
%
%\begin{Exercice}Construire un repère orthogonal.
%\begin{enumerate}
%\item  Placer les points suivants.
%\begin{itemize}
% \item $A(-2;2)$
% \item $C(3;3)$
% \item $D(4;0)$
% \item $E(-2;0)$ 
% \item $F(2;-2)$ 
%\end{itemize}
%\item Calculer les coordonnées des points $B$, $G$, $H$ et $K$ qui vérifient les relations vectorielles suivantes.
%\begin{itemize}
%\item  $\Vect{AC}+\Vect{AB}=\Vect{AD}$
%\item $\Vect{AG}+\Vect{CD}=\Vect{EF}$
%\item $\Vect{AH}-\Vect{CD}=\Vect{EF}$
%\item $\Vect{KA}+\Vect{KC}=\Vect{AD}$
%\end{itemize} 
%  \end{enumerate}
%  \end{Exercice}
%  \section{Multiplication par un réel}
%
%%61
%\begin{Exercice}\label{2G3_E_reperer_produit}
%\begin{enumerate}
%\item Reproduire la figure.
%\item Construire les vecteurs $\Vec u$, $\Vec v$ et $\Vec w$ tels que:
%\begin{itemize}
%\item $\Vec u=2\Vect{AB}$ 
%\item $\Vec v=-3\Vect{BC}$ 
%\item $\Vec w= 0,5\Vect{AB}$
%\end{itemize}
%\item Lire leurs coordonnées.
%\item Les vérifier par le calcul. 
%\end{enumerate}
%\begin{center}
% \begin{tikzpicture}[general, scale=0.8]
%\draw [quadrillage, step=0.5] (-3,-2) grid (3,1.5);
%\axeOI{-3}{3}
%\axeOJ{-2}{1.5}
%\origineO
%\foreach \x/\y/\P/\N in {2/1/left/A,-2/-1/right/B,1.5/-1.5/left/C} 
%{\pointGraphique{\x}{\y}{\N}{\P}}
%\end{tikzpicture}
%\end{center}
%\begin{Correction}
%~\\
% $\Vec u \left(\begin{array}{c}-8\\-4\end{array}\right)$ 
%  $\Vec v \left(\begin{array}{c}-10,5\\1,5\end{array}\right)$\\
%   $\Vec w \left(\begin{array}{c}-2\\-1\end{array}\right)$ 
%\end{Correction}
%\end{Exercice}
%
%%62
%\begin{Exercice}Même consigne qu'à l'Exercice \ref{2G3_E_reperer_produit} avec:
%\begin{itemize}
%\item $\Vec u=\dfrac{ 3}{ 4}\Vect{BC}$
%\item $\Vec v= -\dfrac{1}{2}\Vect{AB}$
%\item $\Vec w=\dfrac{2}{5}\Vect{AB}$
%\end{itemize}
%\begin{center}
%\begin{tikzpicture}[general, scale=1.6]
%\draw [quadrillage, step=0.25] (-1.5,-1) grid (1.5,1.25);
%\axeOI{-1.5}{1.5}
%\axeOJ{-1}{1.25}
%\origineO
%\foreach \x/\y/\P/\N in {-0.25/-0.5/left/A, 0.75/0.75/right/B, 1.25/-0.75/left/C} 
%{\pointGraphique{\x}{\y}{\N}{\P}}
%\end{tikzpicture}
%\end{center}
%\end{Exercice}
%
%%63
%\begin{Exercice}
%Dans le plan muni d'un repère, le vecteur $\Vec u $ a pour coordonnées $\left(\begin{array}{c}-1\\6\end{array}\right) $.\\ Calculer les coordonnées des vecteurs suivants. 
%\begin{enumerate}
% \item $3\Vec u$
% \item $-4\Vec u$
% \item $\dfrac{2}{3}\Vec u$
% \item $-4,5\Vec u$
%\end{enumerate}
%\end{Exercice}
%
%
%%64
%\begin{Exercice}
% Dans le plan muni d'un repère d'origine $O$, on considère les points $P(-3;-1)$ et $R(2;3)$. \\ Quelles sont les coordonnées du point $N$ qui vérifie l'égalité $\Vect{ON}=4\Vect{PR}$?
%\end{Exercice}
%
%%65
%\begin{Exercice}
% Dans un repère, on considère les points $A$ et $B$ de coordonnées respectives $(3;-4)$ et $(-1; 2)$. Quelles sont les coordonnées de $C$ tel que $\Vect{AB}=-5\Vect{AC}$?
%\end{Exercice}
%%66
%\begin{Exercice}
% Dans un repère, on considère les points suivants: $A\left(\dfrac{  2}{ 9};\dfrac{6}{25}\right)$ et $B\left(-\dfrac{5}{6};\dfrac{9}{20}\right)$.\\
%  Calculer les coordonnées de $C$ tel que $\Vect{AC}=\dfrac{ 15}{2}\Vect{AB}$.
%\end{Exercice}
%
%
%%67
%\begin{Exercice}
% Dans le plan muni d'un repère, on considère les points $E\left(\dfrac{ 5}{ 6};-\dfrac{3}{5}\right)$ et $F\left(-\dfrac{1}{8};\dfrac{7}{10}\right)$.\\
%  Quelles sont les coordonnées du point $G$ pour que \\l'égalité $\Vect{EG}=\dfrac{ 4}{ 7}\Vect{EF}$ soit vérifiée?
%\end{Exercice}
%%68
%\begin{Exercice}
% Dans le plan muni d'un repère, les coordonnées des points $A$, $B$ et $C$ sont respectivement $(3;2)$, $(9;-5)$ et $(-9;16)$. Ces points sont alignés. \\
% Calculer le nombre $k$ tel que $\Vect{AB}=k\Vect{AC}$. 
%\end{Exercice}
%
%%69
%\begin{Exercice}[Points et vecteurs]
% Dans un repère, on considère les points suivants.
% \begin{itemize}
% \item $A(3;-1)$
% \item $B(-5;5)$
% \item $M$ $(-1,2)$
% \end{itemize}
% \begin{enumerate}
%  \item Calculer les coordonnées des vecteurs $\Vect{AB}$ et $\Vect{AM}$.
%  \item Montrer que $\Vect{AM}=\lambda \Vect{AB}$, $\lambda$ réel à déterminer.
%  \item Que peut-on dire du point $M$?
% \end{enumerate}
%\end{Exercice}
%%70
%\begin{Exercice}
%      Soient les points $A(3;-2)$, $B(-1;7)$, $C(2;3)$.
%    \begin{enumerate}
%        \item Calculer les coordonnées de
%2$\Vect{AB}$+$\Vect{BC}$.
%        \item Soit le point $M(x;y)$ tel que
%$\Vect{BM}=2\Vect{AB}+\Vect{BC}$.\\
%            Calculer les coordonnées du point $M$.
%    \end{enumerate}
%\end{Exercice}
%
%%71
%\begin{Exercice}[Construction et calculs]
%    Dans un repère on considère les points suivants.
%    \begin{itemize}
%    \item $A(3;-2)$
%    \item $B(-1;2)$
%    \item $C(2;3)$
%    \end{itemize}
%    \begin{enumerate}
%        \item Construire les vecteurs suivants.
%        \begin{enumerate}
%        	\item $\Vect{AB}$
%        	\item $\Vect{BC}$
%        	\item $2\Vect{AB}$
%        	\item $\dfrac{ 1}{ 2}\Vect{BC}$
%        	\item $2\Vect{BC}-4\Vect{BC}$
%        	\item $\dfrac{ 3}{ 5}\Vect{CA}$
%        \end{enumerate}
%    \item Vérifier leurs coordonnées par le calcul.
%    \end{enumerate}
%\end{Exercice}
%
%%72
% \begin{Exercice}
%     Dans un repère, on considère les points:
%     \begin{itemize}
%     \item $A(-3;2)$
%     \item $B(1;-3)$
%     \item $C(1;2)$
%     \end{itemize}
%     \begin{enumerate}
%         \item Déterminer les coordonnées des vecteurs $\Vect{AB}$ et $\Vect{BC}$.
%         \item Déterminer les coordonnées des vecteurs suivants:
%         	\begin{enumerate}
%             	\item $2\Vect{AB}$
%             	\item $\dfrac{ 1}{ 2}\Vect{BC}$
%             	\item $2\Vect{BC}-4\Vect{BC}$
%             	\item $\dfrac{3}{4}\Vect{AC}$
%             \end{enumerate}
%     \end{enumerate}
% \end{Exercice}
%
%\begin{Exercice}[Avec des racines carrées]
%Dans un repère, on considère les points suivants.
%     \begin{itemize}
%     \item $A(2\sqrt{2};3)$
%     \item $B(2;-\sqrt{2})$
%     \item $C(\sqrt{2};-3)$
%     \item $D(3\sqrt{2};-\sqrt{2})$
%     \end{itemize}
%     Déterminer les coordonnées des vecteurs suivants:
%\begin{enumerate}
%\item $\Vect{AB}$
%\item $\Vect{CA}$
%\item $\Vect{AD}$
%\item $\Vect{BD}$
%\end{enumerate}
%\end{Exercice}
%
%
%\section{Colinéarité}
%%74
%\begin{Exercice}
%\label{2G3_E_colinearite}
%Dans le plan muni d'un repère, les  vecteurs suivants sont-ils colinéaires?
%\begin{enumerate}
%\item $\Vec u\left(\begin{array}{c}-2\\3\end{array}\right)$ et $\Vec v\left(\begin{array}{c}3\\-4,5\end{array}\right)$
%\item  $\Vec s \left(\begin{array}{c} 7\\-2 \end{array} \right)$ et $\Vec t\left(\begin{array}{c} 14\\4 \end{array} \right)$
%\item $\Vec w\left(\begin{array}{c}-2\\3\end{array}\right)$ et $\Vec r\left(\begin{array}{c}3\\-4,5\end{array}\right)$
%\end{enumerate}
%\begin{Correction}
%~\\
%\begin{enumerate}
%\item oui \item non \item oui
%\end{enumerate}
%\end{Correction}
%\end{Exercice}
%
%%75
%\begin{Exercice}Dans un repère orthogonal, placer les points:
%\begin{itemize}
%\item $A(-3;1)$
%\item $B(1;3)$
%\item $C(1,-4)$
%\item $D(7;-1)$
%\end{itemize}
%Les droites suivantes sont-elles parallèles?
% \begin{enumerate}
%  \item $(AB)$ et $(CD)$
%  \item $(AC)$ et $(BD)$
% \end{enumerate}
%\end{Exercice}
%
%%76
%\begin{Exercice}Dans un repère orthogonal, placer les points:
%\begin{itemize}
%\item $A\left(-\dfrac{ 1}{ 3};0\right)$
%\item $B\left(\dfrac{2}{3};\dfrac{1}{3}\right)$
%\item $C\left(\dfrac{4}{3};-1\right)$
%\item $D\left(0;-\dfrac{2}{3}\right)$
%\end{itemize}
%Les droites suivantes sont-elles parallèles?
% \begin{enumerate}
%  \item $(AB)$ et $(CD)$
%  \item $(BC)$ et $(AD)$
% \end{enumerate}
%\end{Exercice}
%
%%77
%\begin{Exercice}[Vérification de parallélisme]
% Proposer un algorithme qui vérifie que les droites $(AB)$ et $(CD)$ sont parallèles à partir des coordonnées des points $A$, $B$, $C$ et $D$ entrées par l'utilisateur. 
%\end{Exercice}
%
%%
%%78
%\begin{Exercice}
% Dans un repère, on considère les points $S$, $E$ et $L$ dont les coordonnées sont respectivement $(2;5)$, $(-4;-3)$ et $(5;9)$. 
% 
% Les points $S$, $E$ et $L$ sont-ils alignés?
% 
% Si oui, quelle égalité vectorielle lie $\Vect{SE}$ et $\Vect{SL}$?
%\end{Exercice}
%%79
%\begin{Exercice}Dans un repère orthogonal, les points $M$, $E$ et $R$ ont pour coordonnées respectives:
%\begin{itemize}
%\item $\left( \dfrac{ 2}{ 3};-\dfrac{3}{8} \right)$
%\item $\left( \dfrac{5}{9};\dfrac{5}{2}\right)$
%\item $\left( -\dfrac{7}{6};\dfrac{7}{6}\right)$
%\end{itemize}
%Les points $M$, $E$ et $R$ sont-ils alignés?\\
%Si oui, quelle égalité vectorielle lie $\Vect{ME}$ et  $\Vect{MR}$?
%\end{Exercice}
%
%%80
%\begin{Exercice}[Vérification d'alignement]
% Proposer un algorithme qui vérifie que les points  $A$, $B$ et  $C$ sont alignés à partir de leurs coordonnées entrées par l'utilisateur. 
%\end{Exercice}
%
%
%%81
%\begin{Exercice}
% Dans le plan muni d'un repère, on considère les vecteurs suivants: $\Vec u \left(\begin{array}{c}2\\4\end{array}\right)$ et $\Vec v \left(\begin{array}{c}-1\\3\end{array}\right)$. 
% Quelles sont les coordonnées du vecteur $\Vec w$ vérifiant l'égalité $\Vec w=2\Vec u -\Vec v$?
%\end{Exercice}
%%82
%\begin{Exercice}Le plan est muni d'un repère. 
% \begin{enumerate}
%  \item Placer les points  $V(-1;-1,5)$, $A(-2;0)$ et $T(5;0)$.
%\item Placer $E$ tel que $\Vect{VA}=\dfrac{ 2}{ 3}\Vect{VE}$.
%\item Placer $U$ tel que $\Vect{TU}$ ait pour coordonnées $\left(\begin{array}{c}-2\\0,5\end{array}\right)$.
%\item Que peut-on dire des droites $(OU)$ et $(ET)$? Justifier. 
% \end{enumerate}
%
%\end{Exercice}
%
%
%
%\begin{Exercice}%83
% Dans un plan muni d'un repère, on place les points $A(1;-2)$, $B(-3;1)$, $C(-17;15)$ et $D(-5;6)$.
% 
% Montrer que $ABCD$ est un trapèze.
%\end{Exercice}
%
%\begin{Exercice}%84
% Dans un plan muni d'un repère, on place les points $A(3;-2)$, $B(-5;4)$ et $C(-2;-1)$. 
% \begin{enumerate}
%  \item Calculer les coordonnées de:
%  \begin{itemize}
%  \item $B'$ milieu de $[AC]$;
%  \item $C'$ milieu de $[AB]$.
%  \end{itemize}
%  \item Prouver que $\Vect{C'B'}=\dfrac{1}{2}\Vect{BC}$.
%  \item Calculer les coordonnées de $G$ vérifiant $\Vect{CG}=\dfrac{2}{3}\Vect{CC'}$.
%  \item Les points $B$, $G$ et $B'$ sont-ils alignés? \\Si oui, déterminer le nombre $k$ tel que $\Vect{BG}=k\Vect{BB'}$.
% \end{enumerate}
% 
%\end{Exercice}
%
%
% \begin{Exercice}%85
%Construire un triangle $ABC$. 
%\begin{enumerate}
%\item Placer les points $M$, $P$ et $N$ :
%\begin{enumerate}
% \item $\Vect{BM}=\Vect{BA}+\Vect{BC}$
% \item  $\Vect{MP}=2\Vect{MA}$
% \item  $\Vect{MN}=2\Vect{MC}$
%\end{enumerate}
% \item Prouver que $\Vect{PN}=2\Vect{PB}$. \\ 
% Que peut-on en déduire pour les points $A$, $B$ et $C$?
%\end{enumerate}
% \end{Exercice}
% 
% \begin{Exercice}%86
%Placer trois points $A$, $B$ et $C$ dans un repère.
%    \begin{enumerate}
%        \item Représenter les vecteurs $\Vec a$ et $\Vec
%b$ tels que :
%\begin{itemize}
%\item $\Vec a=\Vect{BC}+2\Vect{AC}$
%\item $\Vec b=3\Vect{AB}-2\Vect{CB}$
%\end{itemize}
%        \item Placer le point $D$ tel que
%$\Vect{AD}=\Vect{a}+\Vect{AB
%}$.
%        \item Placer le point $E$ tel que
%$\Vect{AE}=\Vect{b}-\Vect{AC}$.
%        \item Prouver que $\Vect{AD}=3\Vect{AC}$. 
%        
%        Que peut-on en déduire pour les points $A$, $C$ et $D$?
%        \item Prouver que $\Vect{AB}=\Vect{CE}$. 
%        
%        Que peut-on en déduire pour le quadrilatère $ABEC$?
%    \end{enumerate}
%\end{Exercice}
%
%%%%%%%%%%%%%%%%%%%%%%%%%%%%%%87
%\begin{Exercice}\label{2G4_midemimamb}Soit I le milieu d'un segment [AB].
%\begin{enumerate}
%\item Que peut-on dire du vecteur $\Vec {IA} + \Vec {IB}$?
%\item Démontrer que $\Vec {MI} = 0,5(\Vec {MA} + \Vec {MB})$ pour tout point M.
%\end{enumerate}
%\end{Exercice} 
%
%%%%%%%%%%%%%%%%%%%%%%%%%%%%%88
%\begin{Exercice}[Le petit chaperon rouge] Le petit chaperon rouge rend visite à sa mère-grand dans
%    les bois. Il doit d'abord se rendre au village pour
%    récupérer un pot de beurre puis  passer par la clairière pour faire un   bouquet    de fleurs.
%    
%   Dans un repère $(O;I,J)$, on a représenté
%    la maison du petit chaperon rouge  par le point $D(-1
%    ; -2)$, le village  par le point $V(2 ; 1)$, la
%    clairière  par le point $C(3 ; 0)$ et enfin la maison
%    de mère-grand  par le point $M(0 ; -3)$.
%    \begin{enumerate}
%        \item Faire une figure qui sera complétée par la suite.
%        \item Calculer les coordonnées des vecteurs de déplacement du
%            petit chaperon rouge: $\Vect{DV}$, $\Vect{VC}$, $\Vect{CM}$ et $\Vect{DM}$.
%    \end{enumerate}
%    \begin{enumerate}
%        \item Calculer  les distances $DV$, $VC$, $CM$ parcourues par le petit chaperon rouge
%            depuis le village jusqu'à la maison de sa mère-grand, ainsi que la distance $DM$ correspondant au trajet direct.
% 
%        \item Montrer que le quadrilatère $DVCM$ sur lequel chemine le chaperon rouge est un  rectangle.
%    \end{enumerate}
%    Le grand méchant loup fait peur au petit chaperon rouge. Afin de sécuriser la forêt, un
%    chasseur part à la recherche de la  tanière du loup. 
%    Une vielle sorcière lui dit qu'elle se situe au point $T$ qui vérifie la relation
%    
%    
%    $\Vect{CT}=2\Vect{CM} - \Vect{VM}
%    + \dfrac{3}{ 2}\Vect{DV}$.
%    
%
%            
%On cherche maintenant les coordonnées du point $T$.
%    \begin{enumerate}
%    \item  Placer le point $T$ sur la figure en laissant les traits de
%            construction apparents.
%      \item Calculer les coordonnées de $\Vect{CT}$.
%        \item En déduire les coordonnées de $T$.
%    \end{enumerate}
%\end{Exercice}
%
%%87           
% \begin{Exercice}[Vers la droite d'Euler]
% $ABC$ est un triangle et $A'$, $B'$, $C'$ sont les milieux
%respectifs des côtés $[BC]$, $[AC]$ et $[AB]$.
%            \begin{enumerate}
%                \item Appliquer la formule établie à l'Exercice \ref{2G4_midemimamb} \\aux vecteurs
%$\Vect{AA'}$, $\Vect{BB'}$ et
%                    $\Vect{CC'}$.
%                \item En déduire que
%$\Vect{AA'}+\Vect{BB'}+\Vect{CC'}=\Vect{
%0}.$
%                \item On note $G$ le centre de gravité de $ABC$.\\
%                    En déduire que
%$\Vect{GA}+\Vect{GB}+\Vect{GC}=\Vect{0}
%$.
%            \end{enumerate}
%\end{Exercice}
\end{document}
