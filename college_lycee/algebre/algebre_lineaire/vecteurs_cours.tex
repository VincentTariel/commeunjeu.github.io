\documentclass{book} 
\usepackage{commeunjeustyle} 

\begin{document}


\chapter*{Vecteurs}
\section{Définition}

\begin{Activite}[Translation d'une télécabine]%abstraction=1,link=Translation,seance=Vecteur associé à une translation 
Une télécabine se déplace le long d'un câble de A vers B :\\
\begin{tikzpicture}
\coordinate (A) at (3,5); 
\coordinate (B) at (9,7); 
\pointCFigure{A}{A}{above};
\pointCFigure{B}{B}{above};
\draw [quadrillage,loosely dashed] (0,0) grid (12,8);
\draw (0,4) -- (12,8);
\draw[epais] (3,5) -- (3,4) -- (4,4) -- (4,1) -- (2,1) -- (2,4) -- (3,4);
\end{tikzpicture}\\
Dessiner ci-dessus la télécabine lorsqu'elle sera arrivée au terminus B.\\
On appelle ce déplacement une $\dots\dots\dots\dots\dots$  de A vers B. 
\begin{Correction}
$$A=B+C$$
\begin{tikzpicture}
\coordinate (A) at (3,5); 
\coordinate (B) at (9,7); 
\pointCFigure{A}{A}{above};
\pointCFigure{B}{B}{above};
\draw [quadrillage,loosely dashed] (0,0) grid (12,8);
\draw (0,4) -- (12,8);
\draw[epais]     (3,5) -- (3,4) -- (4,4) -- (4,1) -- (2,1) -- (2,4) -- (3,4);
\draw[epais,A1] (9,7) -- (9,6) -- (10,6) -- (10,3) -- (8,3) -- (8,6) -- (9,6);
\draw [->, epais, color=colordef] (A)--(B);  
\draw [->, epais, color=colordef] (2,4)--(8,6); 
\draw [->, epais, color=colordef] (4,4)--(10,6); 
\draw [->, epais, color=colordef] (4,1)--(10,3);
\draw [->, epais, color=colordef] (2,1)--(8,3);  
\end{tikzpicture}\\Déplacer une figure par \defi{translation}, c'est faire glisser cette figure sans la faire tourner. Pour décrire ce déplacement, on utilise une flèche (sur la figure en rouge) donnant la \defi{direction}, le \defi{sens} et la \defi{longueur} de ce parcours. Cette flèche est un nouvel outil mathématique appelé \defi{vecteur}.  
\end{Correction}
\end{Activite}
\begin{Definition}[Translation]%seance=Vecteur associé à une translation
Soit $A$ et $B$ deux points du plan. \\
On appelle \defi{translation qui transforme $A$ en $B$} la transformation qui, à tout point $C$ du plan, associe
l'unique point $D$ tel que $ABDC$ est un \defi{parallélogramme} (éventuellement aplati).\\
Le point $D$ est l'\defi{image} du point $C$.
\begin{center}
\begin{tikzpicture}[general, scale=0.4]
\draw [quadrillage] (0,0) grid (11,8);
\coordinate (A) at (1,1); 
\coordinate (B) at (3,5); 
\coordinate (M) at (7,2); 
\coordinate (N) at (9,6);
\coordinate (F) at (10.6, 7);
%\draw [loosely dashed] (A)--(N);
%\draw [loosely  dashed] (B)--(M); 
\draw [->, epais, color=A1] (A)--(B); 
\draw [->, epais, color=A1] (M)--(N);
\draw [dashed,  color=A1] (B)--(N); 
\draw [dashed, color=A1] (A)--(M);  
\pointC{A}{A}{below left}
\pointC{B}{B}{above left}
\pointC{M}{C}{below right}
\pointC{N}{D}{above}
%\draw (N) arc (32:52:3);
%\draw (N) arc (32:12:3);
%\draw (3,2.25) node[rotate=122, color=F1]  {{\boldmath $\approx$}};
%\draw (7,4.75) node[rotate=122, color=F1]  {{\boldmath $\approx$}};
%\draw (4,4.25) node[rotate=53, color=C1]  {{\boldmath $\infty$}};
%\draw (6,2.75) node[rotate=53, color=C1]  {{\boldmath $\infty$}};
\end{tikzpicture}
\end{center}
\end{Definition}


\begin{Remarque}%link=Translation,seance=Vecteur associé à une translation
Une \impo{transformation} sert à \impo{modéliser} mathématiquement un déplacement.
\begin{itemize}
\item La \impo{symétrie centrale} est la transformation qui modélise le demi-tour. 
\item La \impo{translation} est la transformation qui modélise le glissement rectiligne. Pour la définir, on indique la direction, le sens et la longueur du mouvement. 
\end{itemize}
\end{Remarque}
\begin{Proposition}[Diagonales du parallélogramme]%seance=Vecteur associé à une translation,dep=Translation
 On considère quatre points $A$,   $B$, $C$ et $D$. \\
La translation, qui transforme $A$ en $B$, transforme $C$ en $D$, si et seulement si   $[ AD ]$ et $[ BC ]$ ont même milieu.
\begin{center}
\begin{tikzpicture}[general, scale=0.4]
\draw [quadrillage] (0,0) grid (11,8);
\coordinate (A) at (1,1); 
\coordinate (B) at (3,5); 
\coordinate (M) at (7,2); 
\coordinate (N) at (9,6);
\coordinate (F) at (10.6, 7);
\draw [loosely dashed] (A)--(N);
\draw [loosely  dashed] (B)--(M); 
\draw [->, epais, color=A1] (A)--(B); 
\draw [->, epais, color=A1] (M)--(N); 
\pointC{A}{A}{below left}
\pointC{B}{B}{above left}
\pointC{M}{C}{below right}
\pointC{N}{D}{above}
\draw (3,2.25) node[rotate=122, color=F1]  {{$\approx$}};
\draw (7,4.75) node[rotate=122, color=F1]  {{ $\approx$}};
\draw (4,4.25) node[rotate=53, color=C1]  {{ $\infty$}};
\draw (6,2.75) node[rotate=53, color=C1]  {{ $\infty$}};
\end{tikzpicture}
\end{center}
\end{Proposition}
\begin{Demonstration}%seance=Vecteur associé à une translation,link=Diagonales du parallélogramme
C'est la conséquence de la propriété: un quadrilatère est un parallélogramme si et seulement si ses diagonales se coupent en leur milieu.
\end{Demonstration}


\begin{Definition}[Vecteurs]%seance=Vecteur associé à une translation,dep=Translation
A chaque translation est associé un \defi{vecteur}. \\
Pour $A$ et $B$ deux points, le \defi{vecteur} $\Vect{AB}$ est associé à la translation qui transforme $A$ en $B$. \\
Le vecteur $\Vect{AB}$ est défini par :
\begin{enumerate}
\item la \defi{direction} (celle de la droite $(AB)$), 
\item le \defi{ sens} (de $A$ vers $B$)
\item la \defi{ longueur} $AB$. 
\end{enumerate}
$A$ est l'\defi{origine} du vecteur et $B$ son \defi{extrémité}.
\end{Definition}



\begin{Definition}[Égalité entre vecteurs]%seance=Égalité entre vecteurs
Deux vecteurs qui définissent la même translation sont dits \defi{égaux}.\\
Deux vecteurs égaux ont : 
\begin{enumerate}
\item même direction \item même sens \item même longueur.
\end{enumerate}
\begin{center}
\begin{tikzpicture}[general, scale=0.75]
\draw [quadrillage, step=0.5] (-3,0.5) grid (4,3);
\draw [->] (-2,2) --node[above] {$\Vect{AB}$} (1,1);
\draw [->] (-0,2.5) --node[above] {$\Vect{CD}$} (3,1.5);
\draw[color=black] (-2,2) node[above left] {$A$};
\draw[color=black] (1,1) node[below right] {$B$};
\draw[color=black] (-0,2.5) node[above left] {$C$};
\draw[color=black] (3,1.5) node[below right] {$D$};
\end{tikzpicture}\\
$\Vect{AB}=\Vect{CD}$
\end{center}
\end{Definition}



\begin{Propriete}[Parallélogramme]%seance=Égalité entre vecteurs
$\Vect{AB}=\Vect{CD}$ si et seulement si $ABDC$ est un parallélogramme (éventuellement aplati).
\begin{center}
\begin{tikzpicture}[general, scale=0.75]
\draw [quadrillage, step=0.5] (-3,0.5) grid (4,3);
\draw [->] (-2,2) --node[above] {$\Vect{AB}$} (1,1);
\draw [->] (-0,2.5) --node[above] {$\Vect{CD}$} (3,1.5);
\draw [dashed] (-2,2) -- (-0,2.5);
\draw [dashed] (3,1.5) -- (1,1);
\draw[color=black] (-2,2) node[above left] {$A$};
\draw[color=black] (1,1) node[below right] {$B$};
\draw[color=black] (-0,2.5) node[above left] {$C$};
\draw[color=black] (3,1.5) node[below right] {$D$};
\end{tikzpicture}
\end{center}
\end{Propriete}

\begin{Methode}[Construire un vecteur]%seance=Égalité entre vecteurs 
Soit  $A$, $B$ et $C$ trois points non alignés.\\
Pour \defi{placer} le point $D$ tel que $\Vect{CD}=\Vect{AB}$, on construit le parallélogramme $ABDC$.
\begin{center}
\begin{tikzpicture}[general, scale=0.5]
\draw [quadrillage] (-4,-2) grid (5,5);
\draw[dashed] (-3,-1)-- (2,0);
\draw[->, epais] (-3,-1)-- (-1,3);
\draw[dashed] (-1,3)-- (4,4);
\draw[->, epais] (2,0)-- (4,4);
\draw (-3,-1) node[below] {$A$};
\draw (-1,3) node[above] {$B$};
\draw (2,0) node[below] {$C$};
\draw (4,4) node[right] {$D$};
\end{tikzpicture}
\end{center}
\end{Methode}



\begin{Remarque}%seance=Égalité entre vecteurs
Une translation peut être définie par un point quelconque et son translaté.\\
Il existe donc une  \impo{infinité} de vecteurs associés à une translation. Ils sont tous égaux.  \\
Le vecteur choisi pour définir la translation est un \impo{représentant} de tous ces vecteurs. \\
La translation  \impo{ne dépend pas} du représentant choisi pour la définir. On le note souvent $\Vect{u}$.
\end{Remarque}

%\begin{Definition}[Vecteur nul]
%Le vecteur associé à la translation qui transforme un point quelconque en lui-même est le \defi{vecteur nul}, noté $\Vect{0}$.\\ Ainsi, 
%$\Vect{AA} = \Vect{BB}  = \Vect{CC}= \ldots =\Vect{0} $ 
%\end{Definition}
%
%\begin{Definition}[Vecteur opposé]
%Le vecteur $\Vect{BA}$ de la translation qui transforme $B$ en $A$ est appelé \defi{vecteur opposé} à $\Vect{AB}$. 
%\end{Definition}
%\begin{Remarque}
%\begin{itemize}
%\item Le vecteur opposé à $\Vect{AB}$ se note $-\Vect{AB}$ et on a l'égalité $\Vect{BA}=-\Vect{AB}$.
%\item La notation $\overleftarrow{AB}$ n'existe pas.
%\end{itemize}
%\end{Remarque}
%
%\begin{Remarque}Deux vecteurs  \impo{opposés} ont même direction, même longueur mais sont de sens contraires.
%\end{Remarque}
%
%\section{Opérations sur les vecteurs}
%
%\subsection{Additions}
%
%\begin{Propriete}[Enchaînement de translations]
% L'enchaînement de deux translations est également une translation.
%\end{Propriete}
%
%
%\begin{DefinitionProposition}[Relation de Chasles]
%Soit $A$, $B$, $C$ trois points. \\
%L'enchaînement de la translation de vecteur $\Vect{AB}$ puis de la translation de vecteur $\Vect{BC}$  est la translation de vecteur $\Vect{AC}$ et on a : $$\Vect{AB}+\Vect{BC}=\Vect{AC}.$$
%Le vecteur $\Vect{AC}$ est le vecteur \defi{somme}.
%\begin{center}
%\begin{tikzpicture}[general, scale=0.5]
%\draw [quadrillage] (-4,-2) grid (5,5);
%\draw[->,epais] (-3,-1)--node[below]{$\Vect{AB}$}  (2,0);
%\draw[->,epais] (2,0)--node[right]{$\Vect{BC}$} (4,4);
%\draw[->,epais] (-3,-1)--node[above,left]{$\Vect{AB}+\Vect{BC}=\Vect{AC}$} (4,4);
%\draw (-3,-1) node[below] {$A$};
%\draw (2,0) node[below] {$B$};
%\draw (4,4) node[right] {$C$};
%\end{tikzpicture}
%\end{center}
%\end{DefinitionProposition}
%%\begin{Remarque}
%% $\Vect{AB}+\Vect{BA}=\Vect{AA}=\Vect{0$.
%%\end{Remarque}
%%
%%
%%\begin{Propriete}Soit $\Vect{{AB}$ et $\Vect{{CD}$ deux vecteurs. Alors : 
%%\begin{itemize}
%%\item  $\Vect{{AB} +\Vect{{CD}=\Vect{{CD} +\Vect{{AB}$ \item $\Vect{{AB} + \Vect{0 = \Vect{
%%{AB}$
%%\end{itemize}
%%\end{Propriete}
%
%\begin{Proposition}[Diagonale du parallélogramme]
%Soit $A$, $B$, $C$, $D$ quatre points. \\
%$\Vect{AD}=\Vect{AB}+\Vect{AC}$ si et seulement si $ABDC$ est un parallélogramme. 
%\begin{center}
%\begin{tikzpicture}[general, scale=0.5]
%\draw [quadrillage] (-4,-2) grid (5,5);
%\draw[->, epais] (-3,-1)--node[below]{$\Vect{AC}$}  (2,0);
%\draw[->, epais] (-3,-1)--node[left]{$\Vect{AB}$} (-1,3);
%\draw[->, epais] (-3,-1)--node[left]{$\Vect{AD}$} (4,4);
%\draw[dashed] (-1,3)-- (4,4);
%\draw[dashed] (2,0)-- (4,4);
%\draw (-3,-1) node[below] {$A$};
%\draw (-1,3) node[above] {$B$};
%\draw (2,0) node[below] {$C$};
%\draw (4,4) node[right] {$D$};
%\end{tikzpicture}
%\end{center}
%\end{Proposition}
%
%\begin{Demonstration}
%On a :
%$$\begin{aligned}
%\Vect{AD}=&\Vect{AB}+\Vect{AC}&\\
%\Leftrightarrow \Vect{AC}+\Vect{CD}=&\Vect{AB}+\Vect{AC}&\text{d'après la relation de Chasles}\\
%\Leftrightarrow \Vect{CD}=&\Vect{AB}\\
% \Leftrightarrow ABDC \text{est un parallélogramme} 
%\end{aligned}$$
%\end{Demonstration}
%
%
%\begin{Methode}[Construire la somme de deux vecteurs]
%On remplace l'un des deux vecteurs par un représentant: 
%\begin{itemize}
%\item soit de même origine afin d'utiliser la règle du parallélogramme;
%\item soit d'origine l'extrémité de l'autre afin d'utiliser la relation de Chasles.
%\end{itemize}
%%\vspace{-0.5em}
%%\begin{enumerate}
%%\item  Construire un carré $ABCD$ de centre $O$.
%%\item Construire les vecteurs \par
%%$\Vect{u=\Vect{AB}+\Vect{OD}$ et 
%%
%%$\Vect{v=\Vect{AD}+\Vect{OC}$
%%\end{enumerate}
%%Correction :
%%\hfill
%%\begin{tikzpicture}[general]
%%  \draw (0,0) node[below] {$A$};
%%  \draw (2,0) node[below] {$B$};
%%  \draw (2,2) node[above] {$C$};
%%  \draw (0,2) node[above] {$D$};
%%  \draw (1,1) node[right] {$O$};
%%  \draw[color=B1] (0.5,0.5) node[above] {{\boldmath $\Vect{u}$}};
%%  \draw (1,-1) node {Avec la relation};
%%  \draw (1,-1.5) node {de Chasles};
%%  \draw (1,3) node[below, white] {$A$};
%%\draw (0,0)--(2,0)--(2,2)--(0,2)--cycle;
%%\draw (0,0)--(2,2);
%% \draw[->, color=A1, epais] (0,0)--(2,0);
%% \draw[->, color=A1, epais] (1,1)--(0,2);
%% \draw[->, color=A1, epais, dashed, line cap=butt] (2,0)--(1,1);
%% \draw[->, color=B1, epais, line cap=butt] (0,0)--(1,1);
%%\end{tikzpicture} 
%%\hfill
%%\begin{tikzpicture}[general]
%%  \draw (0,0) node[below] {$A$};
%%  \draw (2,0) node[below] {$B$};
%%  \draw (2,2) node[above] {$C$};
%%  \draw (0,2) node[above] {$D$};
%%  \draw (1,1) node[right] {$O$};
%%\draw (0,0)--(2,0)--(2,2)--(0,2)--cycle;
%%\draw[dashed] (1,1)--(1,3)--(0,2);
%%\draw (0,2)--(2,0);
%%\draw[->, color=A1, epais] (0,0)--(0,2);
%%\draw[->, color=A1, epais, line cap=butt] (1,1)--(2,2);
%%\draw[->, color=A1, epais, dashed] (0,0)--(1,1);
%%\draw[->, color=B1, epais] (0,0)--(1,3);
%%\draw[color=B1] (1,3) node[right] {$\Vect{v}$};
%%\draw (1,-1) node {Avec la règle};
%%\draw (1,-1.5) node { du parallélogramme};
%%\end{tikzpicture} 
%\end{Methode}
%
%
%\begin{Remarque} 
%$\Vect{AB}+\Vect{AC}=\Vect{0}$ si et seulement si $A$ est le milieu du segment $[BC]$.
%\end{Remarque}
%
%
%\subsection{Soustraction}
%
%\begin{Definition}[Soustraction] \defi{Soustraire un vecteur}, c'est additionner son opposé.
%\end{Definition}
%
%\begin{Exemple}
%Soit trois points $A$, $B$ et $C$ non alignés. \\
%Donner un représentant du vecteur $\Vect{u} =\Vect{AB}-\Vect{AC}$.\\
%On a :\\
%$\Vect{u}=\Vect{AB}-\Vect{AC}$\\
%$\Vect{u}=\Vect{AB}+\Vect{CA}$\\
%$\Vect{u}=\Vect{CA}+\Vect{AB}$\\
%$\Vect{u}=\Vect{CB}$ en utilisant la relation de Chasles.
%\end{Exemple}
%
%
%
%\section{Coordonnées d'un vecteur}
%
%
%\begin{Definition}[Coordonnées d'un vecteur]
%Dans un repère $(O;I,J)$,  on considère la translation de vecteur $\Vect{u}$ qui translate l'origine $O$ en un point $M$ de coordonnées $(a;b)$. \\
%Les \defi{coordonnées du vecteur} $\Vect{u}$ sont les coordonnées du point  $M$ tel que  $\Vect{u} =\Vect{OM}$.\\
%On note $\Vect{u}\left( \begin{array}{c} a\\b\end{array}\right)$.
%\begin{center}
%\begin{tikzpicture}[general, scale=0.5]
%\draw [quadrillage] (-2,-2) grid (8,9);
%\foreach \x/\y/\P/\N in {6/0/below/a,0/5/left/b} 
%{\draw[color=C1] (\x,\y) node {\small  $+$};
%\draw[color=C1]  (\x,\y) node[\P] {{\boldmath $\N$}};}
%\draw[->,color=black] (-2,0) -- (8,0);
%\draw[->,color=black] (0,-2) -- (0,9);
%\origineO
%\draw [dashed, color=F1, epais] (0,5) -- (6,5);
%\draw [dashed, color=J1, epais] (6,0) -- (6,5);
%\foreach \x/\y/\P/\N in {1/0/below/I,0/1/left/J,6/5/right/M} 
%{\draw (\x,\y) node {\small  $+$};
%\draw (\x,\y) node[\P] {$\N$};}
%\draw [->, color=A1, tres epais, line cap=butt] (0,0) --node[left,above] {{ $\Vect{OM}$}} (6,5);
%\draw [->, color=A1, tres epais, line cap=butt] (1,3) --node[left] {{ $\Vect{u}$}} (7,8);
%\end{tikzpicture}
%\end{center}
%\end{Definition}
%
%
%\begin{Proposition}[Égalité des coordonnées]
% Deux vecteurs sont égaux si et seulement si ces vecteurs ont les mêmes coordonnées.
%\end{Proposition}
%
%
%\begin{Proposition}[Coordonnées de $\Vect{AB}$]
%Dans un repère $(O; I,J)$, les  coordonnées du vecteur $\Vect{AB}$ sont $\left( \begin{array}{c} x_B-x_A\\y_B-y_A\end{array}\right)$.
%\end{Proposition}
%
%\begin{Demonstration}
% Soit $A$, $B$ et $M$ de coordonnées respectives $(x_A; y_A)$, $(x_B; y_B)$ et $(x_M; y_M)$ dans un repère $(O; I, J)$ tels que $\Vect{OM}=\Vect{AB}$ et $OMBA$ est un parallélogramme.\\
%Donc $[AM]$ et $[OB]$ ont même milieu.
%    $\renewcommand{\arraystretch}{2}\left \lbrace \begin{array}{c}
%                      \dfrac{x_A+x_M}{2}=\dfrac{x_B+x_O}{2}\\
%                      \dfrac{y_A+y_M}{2}=\dfrac{y_B+y_O}{2}
%                     \end{array} \right.\renewcommand{\arraystretch}{1}
%$   soit  $\left \lbrace \begin{array}{c}
%                     x_A+x_M=x_B\\
%                    y_A+y_M=y_B
%                     \end{array} \right.
%$ soit $\left \lbrace \begin{array}{c}
%                     x_M=x_B-x_A\\
%                    y_M=y_B-y_A
%                     \end{array} \right.
%$
%\end{Demonstration}
%
%\begin{Methode}[Lire les coordonnées d'un vecteur]
%Lire les coordonnées du vecteur $\Vect{u}$ sur la figure ci-dessous. 
%\begin{center}
%\begin{tikzpicture}[general, yscale=0.6]
%\foreach \a/\b/\c/\d in {0/0/5/0, 0/0.87/5/0.87, 0/2.61/5/2.61, 0/3.48/5/3.48, 0/1.74/1/3.48, 1/0/3/3.48, 2/0/4/3.48, 3/0/5/3.48, 4/0/5/1.74} {\draw[quadrillage] (\a,\b)--(\c, \d);}
%\draw[axe] (0,0)--(2,3.48); 
%\draw[axe] (0,1.74)--(5,1.74);
%\coordinate (I) at (2,1.74); 
%\coordinate (J) at (1.5,2.61); 
%\draw (I) node[rotate=-25] {{\boldmath $/$}} node[below]{$I$};
%\draw (J) node {{\boldmath $-$}} node[left]{$J$};
%\draw[epais, ->, line cap=butt, color=F1] (0.5,2.61)--(3,0);
%\draw (1.25,1.305) node[below] {{\boldmath $\Vect{u}$}};
%\end{tikzpicture}
%\end{center}
%Les coordonnées de $\Vect{u}$ sont $\left(\begin{array}{c}4 \\ -3
%\end{array}\right)
%$ car 
%\begin{center}
%\begin{tikzpicture}[general, yscale=0.6]
%\foreach \a/\b/\c/\d in {0/0/5/0, 0/0.87/5/0.87, 0/2.61/5/2.61, 0/3.48/5/3.48, 0/1.74/1/3.48, 1/0/3/3.48, 2/0/4/3.48, 3/0/5/3.48, 4/0/5/1.74} {\draw[quadrillage] (\a,\b)--(\c, \d);}
%\draw[axe] (0,0)--(2,3.48); 
%\draw[axe] (0,1.74)--(5,1.74);
%\coordinate (I) at (2,1.74); 
%\coordinate (J) at (1.5,2.61); 
%\draw (I) node[rotate=-25] {{\boldmath $/$}} node[below]{$I$};
%\draw (J) node {{\boldmath $-$}} node[above left]{$J$};
%\draw[epais, ->, line cap=butt, color=F1] (0.5,2.61)--(3,0);
%\draw (1.25,1.305) node[below, color=F1] {{\boldmath $\Vect{u}$}};
%\draw[epais, ->, line cap=butt, color=A1] (0.5,2.61)--(4.5,2.61);
%\draw (2.5,2.61) node[above, color=A1] {{\boldmath $+4$}};
%\draw[epais, ->, line cap=butt, color=A1] (4.5,2.61)--(3,0);
%\draw (3.75,1.305) node[right, color=A1] {{\boldmath $-3$}};
%\end{tikzpicture}
%\end{center}
%\end{Methode}
%
% 
%\begin{Methode}[Construire un vecteur à partir de ses coordonnées]
%Dans un repère orthonormé, construire le représentant d'origine $A(6; 2)$ du vecteur $\Vect{u}$ de coordonnées $\left( \begin{array}{c} -4 \\3\end{array}\right)$.  \\
% On a :
%\begin{tikzpicture}[general, scale=0.4]
%\draw [quadrillage] (-2,-2) grid (8,6);
%\axeOI{-2}{8}
%\axeOJ{-2}{6}
%\origineO
%\draw [<-, color=A1, tres epais] (2,5) -- (6,2);
%\draw [<-,dashed, color=F1, epais] (2,2) -- (6,2);
%\draw [<-,dashed, color=J1, epais] (2,5) -- (2,2);
%\draw[color=black, color=F1] (4,2) node[below] {{\boldmath $-4$}};
%\draw[color=black, color=J1] (2,3.5) node[left] {{\boldmath $+3$}};
%\draw[color=black, color=A1] (4.5,4) node {{\boldmath $\Vect{u}$}};
%\pointGraphique{6}{2}{A}{below right}
%\end{tikzpicture}
%\end{Methode}
%
%
%
%\begin{Methode}[Repérer un point défini par une égalité vectorielle]
% Dans un repère orthogonal $(O;I,J)$, on a les points $A(-2;3)$, $B(4;-1)$ et $C(5;3)$. \\Calculer les coordonnées 
% \begin{enumerate}
%  \item du vecteur $\Vect{AB}$;
%  \item du point $D$ tel que $\Vect{AB}=\Vect{CD}$.
% \end{enumerate}
%On a:
% \begin{enumerate}
%  \item Les coordonnées du  vecteur $\Vect{AB}$ sont 
%$\left( \begin{array}{c}
%                    x_B-x_A \\ y_B-y_A
%                   \end{array}
%\right)$ soit $\left( \begin{array}{c}
%                    4-(-2) \\ -1-3
%                   \end{array}
%\right)$.
%Donc $\Vect{AB}$ a pour coordonnées $\left( \begin{array}{c}
%                    6 \\ -4
%                   \end{array}
%\right)$.
%
%\item On cherche $(x_D;y_D)$, les coordonnées du 
%point $D$ tel que $\Vect{AB}=\Vect{CD}$.\\ 
%Or,  si deux vecteurs sont égaux alors ils ont mêmes coordonnées. 
%Donc le couple $(x_D;y_D)$ est la solution du système: 
%
%
%$$\left\lbrace \begin{array}{c}
%                    x_D-x_C=x_B-x_A \\ y_D-y_C=y_B-y_A
%                   \end{array}\right.$$
%soit  $\left \lbrace \begin{array}{l}
%                                                        x_D-5=6 \\y_D-3=-4
%                                                       \end{array}
%\right.$
%soit  $\left \lbrace \begin{array}{l}
%                                                        x_D=6+5=11 \\y_D=-4+3=-1
%                                                       \end{array}
%\right.$
%Les coordonnées du point $D$ sont $(11;-1)$.
% \end{enumerate}
%\end{Methode}
%
%
%
%\begin{Proposition}[Somme de deux vecteurs]
%Si $\Vect{u}$ et $\Vect{v}$ sont deux vecteurs de coordonnées respectives $\left (\begin{array}{c}
%x\\y
%\end{array}\right)$ et 
%$\left(\begin{array}{c}
%x'\\y'
%\end{array}\right)$,
%alors les coordonnées du \propri{vecteur somme}, $\Vect{u}+\Vect{v}$, sont 
%$\left(\begin{array}{c}
%x+x'\\y+y'
%\end{array}\right)$.
%\end{Proposition}
%
%\begin{Methode}[Repérer un point défini par une somme vectorielle]
%Dans un repère orthogonal $(O; I, J)$, on place les points $A(2;3)$, $B(4;-1)$, $C(5;3)$ et $D(-2;-1)$.
%Quelles sont les coordonnées du point $E$ tel que $\Vect{AE}=\Vect{AD}+\Vect{CB}$?\\
%On cherche les coordonnées $(x_E;y_E)$ du point $E$ tel que  $\Vect{AE}=\Vect{AD}+\Vect{CB}$. \\
%Donc le couple $(x_E;y_E)$ est solution du système:
%$\left\lbrace\begin{array}{l}
%x_E-x_A=(x_D-x_A)+(x_B-x_C)\\y_E-y_A=(y_D-y_A)+(y_B-y_C)
%\end{array}\right.$
%soit 
%$\left \lbrace \begin{array}{l}
%x_E-2=(-2-2)+(4-5)\\y_E-3=(-1-3)+(-1-3)
%\end{array}
%\right.$ 
%soit 
%$\left \lbrace \begin{array}{l}
%x_E-2=-5 \\y_E-3=-8
%\end{array}
%\right.$ 
%soit 
%$\left \lbrace \begin{array}{l}
%x_E=-5+2=-3 \\y_E=-8+3=-5
%\end{array}
%\right.$\\
%Les coordonnées du point $E$ sont $(-3;-5)$.
%\end{Methode}
%
%
%
%
%\section{Multiplication par un réel}
%
%
%\begin{Definition}[Multiplication par un réel]
%Soit $\Vect{u}$ un vecteur de coordonnées $(x;y)$ et $\lambda$ un réel. \\
%La \defi{multiplication de $\Vect{u}$ par $\lambda$} est le vecteur \defi{$\lambda\Vect{u}$} de coordonnées \defi{$(\lambda x;\lambda y)$}.
%\end{Definition}
%
%\begin{Methode}[Repérer le produit d'un vecteur par un réel]
% Dans un repère orthogonal, construire le représentant d'origine $A(1; 4)$ du vecteur $-0,5 \Vect{u}$ avec $\Vect{u} \left( \begin{array}{c} 2 \\-3\end{array}\right)$.\\
% On a :\\
%$\Vect{u}$ a pour coordonnées $\left( \begin{array}{c} 2 \\-3\end{array}\right)$. 
%Donc $-0,5 \Vect{u}$ a pour coordonnées $\left( \begin{array}{c} -0,5\times 2 \\-0,5\times (-3)\end{array}\right)$  
%soit $\left( \begin{array}{c} -1 \\1,5\end{array}\right)$. 
%\begin{center}
%\begin{tikzpicture}[general, yscale=0.75, xscale=0.75]
%\draw [quadrillage, xstep=1] (-1,-1) grid (4,7);
%\axeX{-1}{4}{1}
%\axeY{-1}{7}{1}
%\draw [->, color=A1,epais] (1,4) -- (3,1);
%\draw [->, color=C1,epais] (1,4) -- (0,5.5);
%\draw [->,dashed, color=A1, epais](1,4) -- (1,5.5);
%\draw [->,dashed, color=C1, epais] (1,5.5)--(0,5.5);
%\draw[color=C1] (0.5,5.5) node[above] {{\boldmath $-1$}};
%\draw[color=A1] (1,4.75) node[right] {{\boldmath $+1,5$}};
%\draw[color=C1] (0.5,4) node[left] {{\boldmath $-0,5\Vect{u}$}};
%\draw[color=A1] (2,3) node[right] {{\boldmath $\Vect{u}$}};
%\pointGraphique{1}{4}{A}{below}
%\end{tikzpicture}
%\end{center}
%\end{Methode}
%
%\begin{Proposition}[Sens en fonction du signe de $\lambda$]
%Soient deux vecteurs $\Vect{AB}$ et $\Vect{CD}$ et $\lambda$ un réel tels que  $\Vect{AB}=\lambda\Vect{CD}$. 
%\begin{itemize}
% \item si $\lambda>0$, $\Vect{AB}$ et $\Vect{CD}$ sont de même sens.
%\item si $\lambda<0$, $\Vect{AB}$ et $\Vect{CD}$ sont de sens contraires.
%\end{itemize}
%\end{Proposition}
%
%\begin{Remarque}
% $\Vect{u}$ et $\lambda \Vect{u}$ ont la même direction.  Leurs sens et leurs longueurs dépendent de $\lambda$.
%\end{Remarque}
%
%
%\section{Colinéarité}
%
%
%\begin{Definition}[Colinéaire]
%On dit que deux vecteurs non nuls sont \defi{colinéaires} si  leurs coordonnées dans un même repère sont proportionnelles. 
%\end{Definition}
%
% \begin{Remarque}
%Par convention, le vecteur nul est colinéaire à tout vecteur $\Vect{u}$. En effet, $\Vect{0}=0.\Vect{u}$.
% \end{Remarque}
%
%
%
%\begin{Proposition}
%Deux vecteurs $\Vect{u}$ et $\Vect{v}$ non nuls sont colinéaires lorsqu'il existe un réel $\lambda$ tel que $\Vect{v}=\lambda\Vect{u}$.
%\end{Proposition}
% 
%
% 
%\begin{Methode}[Vérifier la colinéarité de deux vecteurs]
%Pour vérifier que  deux vecteurs non nuls $\Vect{u}\left (\begin{array}{c}
%x\\y
%\end{array}\right)$ et $\Vect{v}\left(\begin{array}{c}
%x'\\y'
%\end{array}\right)$  sont colinéaires, il suffit de:
%\begin{enumerate}
%\item trouver un réel $\lambda$ non nul tel que $x'=\lambda x$ et $y'=\lambda y$; 
%\item vérifier que les produits en croix, $xy'$  et $x'y$, sont égaux.
%\end{enumerate}
%
%Soit $(O; I, J)$ un repère orthogonal. Les vecteurs suivants sont-ils colinéaires? 
%\begin{enumerate}
%\item $\Vect{u} \left( \begin{array}{c}
%2\\6 
%\end{array}
%\right)$ et $\Vect{v} \left( \begin{array}{c}
%-6 \\ -18
%\end{array}
%\right)$. 
%\item $\Vect{w} \left( \begin{array}{c}
%-5 \\ 3\end{array}\right)$ et $\Vect{z} \left( \begin{array}{c}12 \\ -7\end{array}\right)$.
%\end{enumerate}
%On a :
%\begin{enumerate}
%\item $-6=-3\times 2$ et $-18=-3\times 6$ donc $\Vect{v}=-3\Vect{u}$.
% $\Vect{u}$ et $\Vect{v}$ sont donc colinéaires.
%\item $-5\times(-7)=35$ et $3\times 12=36$.  Les produits en croix ne sont pas égaux.  
%  Donc $\Vect{w}$ et $\Vect{z}$ ne sont pas colinéaires.
%\end{enumerate}
%\end{Methode}
%
%
%\begin{Proposition}[Caractérisation vectorielle des droites parallèles]
%\begin{itemize}
%\item Deux droites $(AB)$ et $(CD)$ sont \propri{parallèles} si et seulement si les vecteurs $\Vect{AB}$ et $\Vect{CD}$ sont colinéaires;
%\item Trois points $A$, $B$ et $C$ sont \propri{alignés} si et seulement si les vecteurs $\Vect{AB}$ et $\Vect{AC}$ sont colinéaires.
%\end{itemize}
%\end{Proposition}
%\begin{Exemple}
%Les points $A(1;2 )$, $B(3;1 )$ et $C(5; 3)$  sont-ils alignés ?\\
%On calcule les coordonnées des vecteurs $\Vect{AB}$ et $\Vect{AC}$ puis les produits en croix :
%$$(x_B-x_A)(y_C-y_A)=(3-1)(3-2)=2\times 1=2.$$
%et ainsi : \\
%$$(y_B-y_A)(x_C-x_A)=(1-2)(5-1)=-1\times4=-4.$$
%Les coordonnées des vecteurs $\Vect{AB}$ et $\Vect{AC}$ ne sont pas proportionnelles. Donc $A$, $B$, $C$ ne sont pas alignés.
%\end{Exemple}

\end{document}
