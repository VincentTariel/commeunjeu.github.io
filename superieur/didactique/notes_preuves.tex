\documentclass{book}
\usepackage{commeunjeustyle}

\begin{document}

\chapter*{Transposition didactique}
\begin{Texte}
\textit{- S'il vous plaît... dessine-moi un mouton...\\
Quand le mystère est trop impressionnant, on n'ose pas désobéir. Aussi absurde que cela me semblât à mille milles de tous les endroits habités et en danger de mort, je sortis de ma poche une feuille de papier et un stylographe. Mais je me rappelai alors que j'avais surtout étudié la géographie, l'histoire, le calcul et la grammaire et je dis au petit bonhomme (avec un peu de mauvaise humeur) que je ne savais pas dessiner. Il me répondit :\\
- Ça ne fait rien. Dessine-moi un mouton.\\
Comme je n'avais jamais dessiné un mouton je refis, pour lui, l'un des deux seuls dessins dont j'étais capable. Celui du boa fermé. Et je fus stupéfait d'entendre le petit bonhomme me répondre :\\
- Non ! Non ! je ne veux pas d'un éléphant dans un boa. Un boa c'est très dangereux, et un éléphant c'est très encombrant. Chez moi c'est tout petit. J'ai besoin d'un mouton. Dessine-moi un mouton.\\
Alors j'ai dessiné.\\
Il regarda attentivement, puis\\
\begin{center}
\includegraphics[scale=0.5]{mouton.jpg}\\
\end{center}
- Non! Celui-là est déjà très malade. Fais-en un autre.\\
Je dessinai :\\
\begin{center}
\includegraphics[scale=0.5]{mouton2.jpg}\\
\end{center}
Mon ami sourit gentiment, avec indulgence :\\
- Tu vois bien... ce n'est pas un mouton, c'est un bélier. Il a des cornes...\\
Je refis donc encore mon dessin :\\
\begin{center}
\includegraphics[scale=0.5]{mouton3.jpg}\\
\end{center}
Mais il fut refusé, comme les précédents :\\
- Celui-là est trop vieux. je veux un mouton qui vive longtemps.\\
Alors, faute de patience, comme j'avais hâte de commencer le démontage de mon moteur, je griffonnai ce dessin-ci.\\
\begin{center}
\includegraphics[scale=0.5]{mouton4.jpg}\\
\end{center}
Et je lançai :\\
- ça c'est la caisse. Le mouton que tu veux est dedans.\\
Mais je fus bien surpris de voir s'illuminer le visage de mon jeune juge :\\
- C'est tout à fait comme ça que je le voulais ! Crois-tu qu'il faille beaucoup d'herbe à ce mouton ?\\
- Pourquoi ?\\
- Parce que chez moi c'est tout petit...\\
- Ça suffira sûrement. je t'ai donné un tout petit mouton.\\
Il pencha la tête vers le dessin\\
- Pas si petit que ça... Tiens Il s'est endormi... 
}

Imaginez que cet extrait du livre "le Petit Prince" soit le compte rendu de l'activité de découverte fait par l'enseignant le Petit-Prince et l'élève Saint-Exupéry et que l'objectif de la séance soit de travailler sur le caractère relatif des représentations.\\ 
A votre avis, quelle est la situation problème posé à l'élève ? Quelles sont les erreurs prévisibles et leurs rôles ? L'élève construit-il ou répète-il une connaissance ?\\
Aussi, cet extrait illustre bien l'importance des différences de représentations entre deux individus et en particulier entre l'enseignant et l'élève. Par exemple, en début de cinquième, un élève se représente un nombre relatif à l'aide d'une température positive et négative. A partir de cette représentation, est-il possible de donner du sens à l'addition ou à la multiplication de nombres relatifs ? Quelles sont les méthodes pour amener l'élève à construire une nouvelle 
représentation d'une notion à partir d'existantes ?\\   
Le but de ce cours est de vous apporter des réponses à ces questions.
\end{Texte}



\section{Généralités}
\begin{Definition}[Savoir savant]Un \defi{savoir savant} est un  corpus  qui  s’enrichit  sans  cesse  de  connaissances  nouvelles,  reconnues  comme  pertinentes  et  valides  par  la  communauté  scientifique  spécialisée.
\end{Definition}
\begin{Exemple}[Éléments de mathématique du groupe Nicolas Bourbaki]
Ce traité est une théorie mathématique classique avec une fondation à 2 étages permettant le raisonnement déductif : les axiomes de la théorie des ensembles de Zermelo-Fraenkel et la logique c'est à dire la machine à générer des preuves. A partir de ces deux étages, l'ensemble des  objets mathématiques courants  est construit  par exemple : "axiomes des ensembles"$\rightarrow\N\rightarrow\Z\rightarrow\Q\rightarrow\R$.
\end{Exemple}
\begin{Definition}[Savoir à enseigner]Un \defi{savoir à enseigner}  est un corpus de  contenus, de normes, des méthode à transmettre à un groupe d'apprenants.    
\end{Definition}
\begin{Exemple}[Programme de l'éducation nationale]
Le programme de l'éducation nationale est un savoir à enseigner décrits, précisés, dans l’ensemble des textes "officiels" dans les programmes, les instructions officielles, les attendus, les livrets d'accompagnements.  Par exemple avec la "réforme des maths modernes" fortement inspirée par éléments de mathématique du groupe Nicolas Bourbaki, le programme  se fonde autour de la notion de structure :
\begin{itemize}
\item primaire : théorie des ensembles
\item collège : relations, applications. Par exemple,
\begin{Definition}[Droite  réelle (en quatrième en 1971)]
Un  ensemble  $D$  d’éléments  appelés  points  est  une  \defi{droite  réelle}, s’il  existe  une  famille  de  bijections  de  $D$  sur  l’ensemble  des  nombres réels, appelés graduations de $D,$ vérifiant l’axiome suivant :
\begin{itemize}
\item pour deux graduations quelconques $g$ et $g’$ de la même droite réelle $D$, il existe deux nombres réels $a$ et $b$, tels que pour tout point $M$ de $D$, $g’(M) = a.g(M) + b$.
\end{itemize}
Le nombre réel $g(M)$ est appelé \defi{abscisse} dans la graduation $g$ du point $M$.
\end{Definition}
\item lycée : groupes, corps, espace vectoriel, géométrie affine.
\end{itemize}
La présentation axiomatique classique des mathématiques "paraît merveilleusement adaptée à l'enseignement. Elle permet à chaque instant de définir les objets que l'on étudie à l'aide des notions précédemment introduites et, ainsi, d'organiser l'acquisition de nouveaux savoirs à l'aide des acquisitions antérieures"
\end{Exemple}
\begin{Definition}[Savoir enseignés]Un \defi{savoir enseigné}  est construit par l'enseignant et mis en \oe uvre dans la classe.
\end{Definition}
\begin{Definition}[Savoir appris]Un \defi{savoir appris} est l’ensemble des savoirs acquis par un apprenant.
\end{Definition}
\begin{Definition}[Transposition didactique]
La \defi{transposition didactique} est la transformation  d'un savoir savant au savoir enseigné.\\
Ces transformations, appelées "transposition didactique" va se faire en deux étapes : la première est celle qui va faire passer le savoir savant au savoir à enseigner, cette transposition externe conduit à la définition des programmes d'enseignement de chaque discipline scolaire et la deuxième, la transposition interne est celle qui fait passer ce savoir à enseigner, au savoir réellement enseigné, cette transposition est celle que fait chaque enseignant dans ses classes en fonction de ses élèves et des contraintes qui lui sont imposées (temps, examens, conformité à des canons scolaires établis etc.).\\
\begin{center}
\begin{tikzpicture}
 \node   (sv) {Savoirs savants};
 \node [ below=1.5cm of sv] (sae) {Savoir à enseigner};
 \node [below= 1.5cm of sae]  (se) {Savoir enseigné };
  \node [below= 1.5cm of se]  (sa) {Savoir appris };
 \draw[-stealth] (sv.south) -- (sae.north)
    node[midway,right]{Transposition didactique externe};
  \draw[-stealth] (sae.south) -- (se.north)
    node[midway,right]{Transposition didactique interne};
   \draw[-stealth] (se.south) -- (sa.north);

\end{tikzpicture}
\end{center}
\end{Definition}

\subsection{Optimisation du savoir}



\begin{Definition}[Optimisation du savoir]
Éviter tout dogmatique meme justifier par la statistique : "90 pourcents des élèves apprennent à lire en 3 mois avec la méthode globale"   
Complexe : c'est la meilleur méthode !!!!



Eviter à tout prix  : "je suis nul en maths !"
"Là où il y a une volonté, il y a un chemin."
"Choisis toujours le chemin qui semble le meilleur même s'il paraît plus difficile : l'habitude le rendra bientôt agréable."
"La perfection est un chemin, non une fin."
"Il vaut mieux suivre le bon chemin en boitant que le mauvais d'un pas ferme."
"Ce n'est pas parce qu'un chemin est en impasse qu'il ne conduit nul part."
"Sur le chemin, l'imprévu."
Curiosité de l'intelligence 

Qu'est ce que je veux apprendre ? Pour un concours, comme outil pour faire du machine learning, comprendre, boulimique ?
Comment fonctionne mon cerveau ?
 

Difficile question : 
compréhension et la résolution de problèmes, apprentissage des procédures
Quantification ? Court et long terme
\end{Definition}


\section{Transposition externe}


Pas le choix du savoir à enseigner en fonction de l'étudiant avant la première.\\
En 6 ième des élèves 
Différentiation dans l'acquisition de nouveaux savoirs...

Quid de l'individualisation ?\\

Vision utilitarisme -> maths comme un outil etc 



Parler de l'algèbre linéaire
Plusiseurs chemins



\begin{Exemple}[Transposition didactique]
\begin{enumerate}
\item Savoir savant : livre de Y. Chevalard  "La Transposition didactique: du savoir savant au savoir enseigné", livre   de   J.L. Dorier, G. Gueudet  "Enseigner les mathématiques, Didactique et enjeux de l'apprentissage" 
\item Savoir à enseigner : destiné aux Meef Mathématiques les attendus théoriques de ce cours d'introduction se limite à quelques définitions (analyse didactique a priori, compétence, concept, consigne, contrat didactique, contrat , dévolution du problème, erreur, institutionnalisation, objectif, problématique, procédure, procédure experte, séance, séquence d'apprentissage, situation problème, système didactique, tâche, transposition didactique, théories d'apprentissages, validation, variable didactique)  
\end{enumerate}
\end{Exemple}


\section{Transposition interne}


Certains uniquement des procédures, d'autres d'abord des procédures puis de l'abstraction, 
D'autres la compréhension


Niveau d'abstraction




Groupe d'apprenant différentiation 


Temporalité




Savoir à enseigner :




Savoir savant
Gauss : 
« lorsqu’un bel édifice est achevé, on ne doit pas y lire ce que fut l’échafaudage ».










Algèbre linéaire :
Cours sur les groupes, les anneaux et les corps
Cours sur les espace vectoriels




Savoir à enseigner


Savoir enseigné




\chapter*{Transposition didactique de la preuve}



\section{Logique formelle}
\subsection{Connecteurs}
\subsection{Ensembles et quantificateurs}

\section{Algorithme}
\subsection{Introduction-Conclusion}
\subsection{Raisonnements}
\subsection{Principes}
\subsubsection{Ponts}
\subsubsection{Vice et versa}
\subsubsection{Changement de cadre}





\section{Preuve évolutive}
\begin{Proposition}
Étant donnée une matrice $A$ carré.\\
A est inversible si et seulement si  A est inversible à gauche ou A est inversible à droite.
\end{Proposition}
Supposons qu'il existe $B\in\MnK$ tel que $AB=I_n$.\\
Plusieurs éléments de preuves :
\begin{itemize}
\item \impo{Espace vectoriel} :  comme $AB=I_n$, les colonnes de $A$, notées $C_1,\dots,C_n$ forment une famille libre de $\K^n$ car $\rg(A)\geq \rg(AB)$, donc une base car dimension finie. Soit $e_1,\dots,e_n$ la base canonique de $\K^n$ notés en colonnes. Comme les colonnes de $A$ forment une famille génératrice, il existe des coefficients $b'_{ij}\in\K$ tels que :
$$ e_j=\sum_{j=1}^n b'_{ij} C_j,\quad \forall i\in\{1,\dots,n\}.$$
Or les vecteurs de la base canonique sont précisément les vecteurs colonnes de la matrice identité. En posant $B'=(b_{ij})\in\MnK$, il en résulte que $B'A=I_n$.\\
$B=I_nA=(B'A)B=B'(AB)=B'I_n=B'$ donc  $B'A=I_n$. 
\item \impo{Système linéaire} :  on considère l'équation $BC=aI_n$, d'inconnues $C\in\MnK$ et $a\in\K$. Elle se traduit par un système linéaire de $n^2$ équations à $n^2+1$ inconnues. Ce système admet donc une solution non triviale $(C_0,a_0)$. On a $a_0\neq $. En effet sinon on aurait $C_0=ABC_0=a_0A=0$ ce qui voudrait dire que $(C_0,a_0)$ est triviale.\\
$C=a_0^{-1}C_0$ vérifie $BC=I_n.$
\item \impo{Application linéaire} soit $f$ et $g$ les endomorphismes canoniquement associés aux matrices $A$ et $B$. Comme $fg=Id_\K^n$, $fg$ est injectif donc $g$ aussi. Comme la dimension est finie,  $g$ est bijective d'inverse $f$. D'où $gf=Id_\K^n$ et $BA=I_n.$
\item \impo{Déterminant} : $\det(A)\det(B)=\det(AB)=\det(I_n)=1$. Donc  $\det(A)\neq 0$ et $A$ est inversible.
\item \impo{Polynôme annulateur} : 
Soit $P=a_0+a_1X +\dots+a_pX^p$ un polynôme annulateur de $A$. Comme $P$ est non nul, il existe $i\in\{0,\dots,p\}$ tel que   $a_i\neq 0$ et $a_j=0$ pour $j\in \{0,\dots,i-1\}$.
En multipliant par $B$ à droite $i$ fois, on peut supposer que $a_0$ est non nul, on obtient un polynôme $P(a)$, tel que $AP(A)=I_n$, et comme $P(A)$ commute avec $A$, $P(A)A=I_n$ également.\\
Ainsi $A$ est inversible à droite et à gauche, et donc $B=P(A)$, et $AB=I_n$.
\end{itemize}





\subsection{Indications}
\subsection{Introduction-Conclusion}
\subsection{Raisonnements}
\subsection{Principes}




\end{document}
